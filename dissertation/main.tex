\documentclass[11pt,oneside,openany,a4paper, %... Global layout
afrikaans,UKenglish, %... Global language options
PhD, a5block %... stb-thesis options
]{stb-thesis}


\usepackage[english]{babel}
\usepackage{graphicx}
\usepackage[utf8]{inputenc}
\usepackage[onehalfspacing]{setspace}
\usepackage{color}
\usepackage{parskip}%......... Paragraph spacing + zero indent
\usepackage{amssymb} % only if necessary
\usepackage{bm} %.. Bold math characters (after fonts)
\usepackage{hyperref}

\usepackage[numbers]{stb-bib}%........................... Bibliography
%.......... Auhor-year style
\addto{\captionsafrikaans}{\renewcommand{\bibname}{Lys van Verwysings}}
\addto{\captionsUKenglish}{\renewcommand{\bibname}{List of References}}

% \iftrue
% \usepackage[backref=page]{hyperref}%.... Hyperlinks & backreferences
% \else
% \usepackage[pageref]{backref}%........... Citation backreferences
% \fi
% \renewcommand*{\backrefalt}[4]{%s
% \ifcase #1 (Not cited.)%
% \or (Cited on page~#2.)%
% \else (Cited on pages~#2.)%
% \fi}

\newcommand\articleinfo[4]{%
\noindent\rule{\textwidth}{0.5pt}
\begin{center}\small\textit{#1}\par
    \begin{spacing}{0.1}
    \small Accepted in #2 on #3\par
    \small doi: #4\noindent
\end{spacing}\noindent
\end{center}\noindent
\noindent\rule{\textwidth}{0.25pt}%
}

\usepackage{paper1}
\usepackage{paper2}
\usepackage{paper3}

% Subappendices
\usepackage{appendix}
% Start of subappendices environment
\AtBeginEnvironment{subappendices}{%
\section*{Appendices}
\addcontentsline{toc}{section}{Appendices}
}

% End of subappendices environment
\AtEndEnvironment{subappendices}{%
}

% Chapter format
\usepackage{titlesec}
\titleformat{\chapter}
  {\LARGE\bfseries}{\thechapter}{1em}{}
\titlespacing*{\chapter}{0pt}{3.5ex plus 1ex minus .2ex}{2.3ex plus .2ex}


\graphicspath{{./content/paper2/fig/}{./content/paper1/fig/}}

\usepackage{iftex}
\ifxetex
\usepackage[math-style=TeX, bold-style=TeX]{unicode-math}
\setmainfont{Cambria}%........................ (Windows)
\setsansfont[Scale=MatchLowercase]{Calibri}
\setmonofont[Scale=MatchLowercase]{Consolas}
\setmathfont{Cambria Math}
\defaultfontfeatures{Ligatures=TeX}
\let\bm\symbfit%.............................. Bold math
\else
\usepackage{textcomp}%........................ Additional text symbols
\usepackage[T1]{fontenc}%..................... Type 1 outline fonts
\usepackage{bm}%.............................. Bold math fonts
\fi

\begin{document}
\title{\AorE{Wavelet Scattering Transformasies Toegepas op Walvis Vokaliserings}{Wavelet Scattering Transforms Applied to Whale Vocalisations}}
\author{M.W. Rademan}{Marco Wiehann Rademan}
\faculty{Electrical and Electronic Engineering}
\degree{PhD}{Doctor of Philosophy}
\supervisor{D.J.J. Versfeld}
\cosupervisor{J.A. du Preez}
% \date{August}{2024} 
\SetCopyrightOff
\address{\AorE{Departement Elektriese en Elektroniese Ingenieurswese,\\ Stellenbosch Universiteit,\\
Privaatsak X1, 7602 Matieland, Suid Afrika.
}{%
Department of Electrical and Electronic Engineering,\\ Stellenbosch University,\\
Private Bag X1, 7602 Matieland, South Africa.}}


% \frontmatter
\begin{singlespace}
\TitlePage%..................... Single spaced title page
\end{singlespace}


\begin{singlespace}
\DeclarationDate{\today}
\DeclarationPage
\end{singlespace}
% \DeclarationPage


\chapter*{Declaration of Publications}

This dissertation includes three original papers published in peer-reviewed journals. The writing of these papers were the responsibility of myself and contain original work I have individually developed, under the supervision of professors Versfeld and Du Preez. Each paper includes a detailed statement of author contributions using the CRediT system where they appear, along with all necessary information to locate the original publications.

\textit{Hierdie proefskrif sluit drie publikasies in, gepubliseer in eweknie-geëvalueerde joernale. Die skryf van hierdie publikasies was die hoofverantwoordelik van myself, waarvan ek die inhoud individueel ontwikkel het onder die leierskap van my toesighouers, Professore Versfeld en Du Preez. 'n Deeglike bydraestelling word voorsien waar die publikasies in hierdie proefskrif verskyn in terme van die CRediT sisteem.}

\chapter*{Acknowledgements}

I would like to thank my supervisors, for encouraging me to upgrade my Master's degree to PhD, allowing me to pursue my love for academics. It is an honour to be able contribute something to a world filled with brilliant minds. 

A special thank you to my parents, who have supported me in various ways in my entire academic career. Without them, I would not have made it this far. A kind thank you to my mother, who has stood by side, and my father, who has helped me through writing countless impactful abstracts and dedicated many hours of his time to academic discussions. 

Finally, a heartfelt thanks to my partner, who has stood by me through it all and meticulously read every single word. My countless work hours and your endless patience have finally paid off. I love you dearly.

\begin{abstract}
    
In this dissertation by publication, we investigate the application of wavelet scattering transforms on whale vocalisations, which is presented as three published papers.

We first consider how wavelet transforms can provide an alternative time-frequency (TF) decomposition for use in signal detectors. Detectors typically utilise the short-time Fourier transform, for which the continuous wavelet transform (CWT) is a suitable alternative. As an additional contribution, we improve upon the spectral entropy (SE)detector, which is known to be superior in terms of signal detection compared to energy detectors.

In the second publication, we expand on the relatedness of the CWT to wavelet scattering. Wavelet scattering is a feature extraction method akin to the filters of a convolutional neural network (CNN) frontend and Mel-frequency cepstral coefficients (MFCCs). We show that wavelet scattering coefficients can be combined with a SE detector, which can be fed into a classification system. We further improve upon the SE detector presented in the first paper and provide a critical discussion of the challenges faced when evaluating performance of the proposed methods on a large real-world dataset. 

Finally, the concepts established in the previous papers are used to construct an entirely new and generalised form of the wavelet scattering transform -- separable wavelet scattering. We extend the definition of one-dimensional scattering to an arbitrary number of signal dimensions. Separable scattering benefits from various computational and conceptual advantages, which has not been generalised to this extent in the current literature.


    
\end{abstract}

\begin{abstract}[afrikaans]
    
In hierdie publikasie proefskrif, ondersoek ons die toepassing van \textit{wavelet} verstrooiing transformasies op walvis vokaliserings, deur middel van drie gepubliseerde joernaal arikels.

Eerste oorweeg on hoe \textit{wavelet} transformasies 'n alternatiewe tyd-frekwensie (TF) ont-binding vir seindetektors kan voorsien. Detektors maak gewoonlik gebruik van die kort-tyd Fourier transform (STFT) waarvoor die kontinu\"e \textit{wavelet} trasnform (CWT) 'n alternatief is. Ons verbeter ook die spektrale entropy (SE) detector, wetend dat dit beter presteer as energie-gebasseerde seindetektors.

In die tweede publikasie brei ons uit op die verwantskap tussen die CWT en \textit{wavelet} verstrooiing. \textit{Wavelet} verstrooiing is metode wat kenmerke uit seine onttrek in 'n soorgelyke manier aan konvolusionele neurale netwerke (CNNs) en Mel-frekwensie kepstrale ko\"effisi-\"ente (MFCCs). Ons wys dat \textnormal{wavelet} verstrooiing met 'n SE detektor gekombineer kan word wat verder in 'n klasifikasiesisteem invloei. Ons bied 'n kritiese gesprek aan wat die uitdagings van sisteemevaluasie deur middle van 'n groot praktiese datastel bespreek.

Laastens word die konsepte wat in die vorige publikasies vasgestel is gerbuik om 'n splinter-nuwe en veralgemene \textnormal{wavelet} verstrooiing metode te definieer -- skeibare \textnormal{wavelet} ver-strooiing. Ons veralgemeen die eendimensionele \textnormal{wavelet} verstrooiing to 'n arbitr\^ere aantal intree-dimensies wat nog nie tevore in die bestaande literatuur ondersoek is nie. 
    
\end{abstract}

\tableofcontents
\listoffigures
\listoftables
\chapter*{Nomenclature}
\label{chap:nomenclature}
\markboth{Nomenclature}{\uppercase{Nomenclature}}
\addcontentsline{toc}{chapter}{\nameref{chap:nomenclature}}
\section*{Notation}
Some notations are used interchangably, which is obvious from the context. Due to the nature of the publication format, notation is additionally explained within the article it occurs.

\begin{tabular}{p{0.15\textwidth} p{0.75\textwidth}}
    $f(x)$ & a continuous function of $x$ \\
    $f(x, y)$ & a continuous multivariable function of $x$ and $y$ \\
    $f[n]$ & a discrete-time (sampled) function indexed by $n$ \\
    $f[n, m]$ & a multi-variable discrete-time function indexed by $n$ and $m$ \\
    $\mathcal{F} \{f\}$ & the Fourier transform operator acting on $f$\\
    $\hat{f}$ & the Fourier transform of $f$ or an estimate of $f$, derived from context\\
    $x^*$ & the complex conjugate of $x$\\
    $p(x)$ & a probability density \\
    $p(x|y)$ & a conditional probability density \\
    $\{a, b, ...\}$ & a set of numbers\\
    $A \setminus B$ & set minus: the set $A$ excluding the elements of $B$\\
    $A \times B$ & Cartesian product of sets\\
    $| x |$ & magnitude of a number (modulus), or size of a set\\
    $\mathbb{R}$ & the set of real numbers\\
    $\mathbb{C}$ & the set of complex numbers\\
    $\mathbb{N^+}$ & the set of natural numbers (excluding 0)\\
    $\mathbb{Z}$ & the set of integers\\
    $x \sim p$ & $x$ is drawn from the probability density $p$\\
    $\frac{d}{dx} f$ & derivative with respect to $x$ of a function $f$\\
    $\mathcal{U}_j x$ & scalogram of the $j$'th level of $x$\\
    $\mathcal{S}_j x$ & scattering coefficients of the $j$'th level of $x$\\
    $x * y$ & convolution of two functions\\
    $x \otimes y$ & circular convolution of two discrete functions\\
    $(x)_{\downarrow r}$ & a discrete function subsampled by a factor $r$\\
    $\mathbf{x}$ & a vector\\
\end{tabular}

\section*{Notation (continued)}
\begin{tabular}{p{0.15\textwidth} p{0.75\textwidth}}
    $(\vect{x})_{\downarrow \vect{r}}$ & a discrete multidimensional signal subsampled in each dimension by the components of $\vect{r}$ \\
    $X[k,m]$ & a time-frequency decomposition of a discrete signal $x$, with $k$ the frequency index and $m$ the time index \\
    $S[k,m]$ & power spectrum calculated from $X[k, m]$ \\
    $H[m]$ & spectral entropy calculated from $S[k, m]$ \\
    $X(s, b)$ & continuous wavelet transform of $x$, with $s$ as scale and $b$ as translation \\
    $\uppsi_\lambda$ & used as short-hand for a wavelet dilation -- $\lambda\uppsi(\lambda t)$\\
    $x \triangleq y$ & $x$ is defined as $y$\\
\end{tabular}


\section*{Acronyms}
\begin{tabular}{p{0.15\textwidth} p{0.75\textwidth}}
    GMM & Gaussian mixture model \\
    DT & discrete-time \\
    POI & point(s) of interest \\
    EM & expectation maximization \\
    SE & spectral entropy \\
    PAM & passive acoustic monitoring \\
    PSD & power spectral density \\
    STFT & short-time Fourier transform \\
    TPR & true positive rate \\
    FPR & false positive rate \\
    PDF & probability density function \\
    SNR & signal-to-noise ratio \\
    AMFM & amplitude and frequency modulated \\
    BLED & band-limited energy detection \\
    TF & time-frequency \\
    CWT & continuous wavelet transform \\
    DWT & discrete wavelet transform \\
    SWT & stationary wavelet transform \\
    WPD & wavelet packet decomposition \\
    MODWT & maximal overlap discrete wavelet transform \\
    MFCC & Mel-frequency cepstral coefficient \\
    DNN & deep neural network \\
    CNN & convolutional neural network \\
    HMM & hidden Markov model \\
    SOTA & state-of-the-art \\
    LSTM & long-short-term memory \\
    EWT & Endangered Wildlife Trust \\
    WS & wavelet scattering \\
    WST & wavelet scattering transform \\
    FFT & fast Fourier transform \\
    GPU & graphics processing unit \\
    ROC & receiver operating characteristic \\
    CLS & cosine-log scattering \\
    LDA & linear discriminant analysis \\
    ML & machine learning \\
    NN & neural network \\
    SVM & support vector machine \\
\end{tabular}

\section*{Acronyms (continued)}
\begin{tabular}{p{0.15\textwidth} p{0.75\textwidth}}

    1D & 1-dimensional \\
    2D & 2-dimensional \\
    SIFT & scale-invariant feature transform \\
    LPF & low-pass filter \\
    IR & impulse response \\
    FIR & finite impulse response \\
    AUC & area under curve \\
    SD & standard deviation \\
    STPR & soft true positive ratio \\
    STNR & soft true negative ratio \\
    BPF & band-pass filter \\

\end{tabular}

% TODO: paper 1 acronyms and consolodate



% \begin{singlespace}
\chapter{Introduction}

\Ac{ml} models have become mainstay in software products and automation pipelines. Often, when there is processing power available, a \ac{ml} model will be deployed to increase efficiency, accuracy and limit human interaction. Various models, whether they are statistical, Bayesian or probabilistic, decision trees or \ac{nn}, are configured to assist or perform various tasks across many industries.

Modern research is very focussed on advancing \ac{ml} integration, improving model performance and investigating new techniques. A large branch of this research is particularly focussed on data-driven approaches using \acp{nn}. The collection of data therefore becomes an important part of model training and deployment. \Acp{nn} present some challenges and concerns, since they require vast amounts of data, and their operation is often unpredictable and not well understood. They are typically treated as ``black box'' functions, which researchers improve and design through trial and error via hyper-parameter optimisation and architecture exploration.

In the field of \ac{pam}, which focuses on monitoring wildlife using non-invasive acoustic sensors, the majority of recent \ac{sota} models are \ac{nn}-based. However, metrics of real-life deployment and practical model evaluation are often ignored, since the acoustical landscape of natural environments can be surprisingly varied. For example, one cannot expect a \ac{nn} model which was not trained on data that contains rainfall noise, to effectively operate when it starts raining. 

Long moments of silence, weather patterns and migrating animals contribute to an ever-changing soundscape which can cause many false positives in naively trained and deployed models. Systems which aim to track specific species are often faced with this ``needle in a haystack'' problem, in which the sounds you are looking for are rare compared to the lifetime of the deployed sensors. As such, human intervention with recorded audio is a must-have when trying to reliably track species via audio signals, which can be a very laborious task indeed.

As such, designing models and systems which are robust to all the aforementioned factors are critical in automating wildlife conservation studies based on non-invasive audio sensors. This leads to the natural question of the suitability of \acp{nn} in this field and to what extent they can be used to be effective.

At the core of many audio-based systems lies a \ac{tf}-decomposition, typically \acp{mfcc} or the \ac{stft}. This dissertation focuses on \ac{tf} decompositions, and how they can be leveraged in order to reduce the usage or even entirely replace \acp{nn} in \ac{pam} systems. At the end of this investigative journey, we naturally arrive an entirely new generalisation of a feature extraction method based on \ac{ct} wavelets, which can be utilised as a drop-in replacement for a \ac{cnn} front-end. The integration and investigation of wavelet-based processing in \ac{pam} systems are relatively scarce, even though it is very prevalent in other fields, which we aim to remedy in the publications presented in this dissertation.

\section{Problem Statement}

In summary, some of the many challenges posed by \ac{pam} systems are:
\begin{enumerate}
    \item Long periods of silence or noise, leading to unbalanced data classes which can have a tendency for creating many false positives in models if not trained correctly.
    \item Constantly evolving noise landscape, with the interference of other bioacoustical sounds which we are often not interested in classifying.
    \item Large audio databases and recordings, which motivates the use of fast algorithms and processing techniques.
    \item Poor \ac{snr}, often less than 0 dB.
    \item Incomplete or mislabeled training data, due to poor \ac{snr} and human error.
    \item Small datasets, which only cover a portion of environment variability, making it unsuitable for a general model.
    \item Non-white, non-stationary background noise, which makes classification and signal detection non-trivial tasks.
    \item Poorly understood species behaviour in terms of their bioacoustics and movements.
\end{enumerate}

All the above factors seem demotivate the use of \acp{nn} for the following reasons: principle of operation can be unpredictable and not well understood; can take long to compute; requires reliable training data or special techniques to mitigate problematic data. Although it is possible to address these challenges, it is worth exploring non-\ac{nn} methods. 

As a reponse to the challenges posed by \ac{pam} systems, we focus specifically on alternative aproaches to \acp{nn}, by considering the modification of traditional classifiers and detectors with wavelet-based \ac{tf}-decompositions.

\section{Layout}

As this is a dissertation by publication, each paper can be read as a stand-alone item, with all the necessary literature and knowledge capture within. As such, this dissertation is structured with additional contextual chapters (chapters \ref*{chap:p1i}, \ref*{chap:p2i}, \ref*{chap:p3i}) prior to each publication (chapters \ref*{chap:p1}, \ref*{chap:p2}, \ref*{chap:p3}).

Each contextual chapter is in itself an introduction to the literature and concepts presented by the paper in the following paper. Contextual chapters serve to unify the content of the published papers and show the natural progression of the analysis and application of \ac{tf} decompositions. All relevant literature is therefore gradually introduced, due to each paper conceptually expanding on the previous paper's ideas and contributions.

Although this dissertation is mostly focussed on the application to \ac{pam} systems, the theory and findings it presents are general. As such, the contextual chapters are generalised mathematical descriptions of the concepts present in each paper, whereas the papers 1 and 2 (chapters \ref*{chap:p1}, \ref*{chap:p2}) are focussed on applying it to \ac{pam} systems. The final paper (chapter \ref*{chap:p3}) shows a generalised result from all the knowledge and investigations presented in prior papers.

Each paper is presented as-is, with no modifications, save for unifying mathematical notation and the renumbering of items.


\section{Journal Papers and Contributions}

\chapter{The Relevance of Time-Frequency Analyses}


\chapter{Separable Wavelet Scattering}
\label{chap:p3}




\articleinfo{M.W. Rademan, D.J.J Versfeld, J.A. Du Preez}{IEEE Signal Processing Letters}{15 May 2024}{doi}{Wavelet scattering is a widely used feature extraction method due to it efficacy in extracting invariant features, 
while retaining any lost high frequency information resulting from averaging to obtain the desired amount of invariance. 
By generalising a 1-dimensional scattering transform, we extend its definition to an arbitrary number of independent dimensions. 
We show that, in a modern deep learning setting, separable wavelet scattering performs similarly to its non-separable counterparts 
with the MNIST hand-written digit dataset. We additionally demonstrate state-of-the-art results for a subset of the MedMNIST3D datasets.}


\section{Introduction}
\Ac{ws} has seen widespread use in classification applications as a powerful feature extraction method. It is an especially effective method for small datasets, since the feature filters are reminiscent of a \ac{cnn} front-end, while requiring no learning. The resulting features are invariant across all averaged dimensions, and exhibit separable class subspaces, allowing linear classifiers to be used with high efficacy \citep{2dscattering}.

Various forms of \ac{ws} exist, with \ac{1d} scattering first proposed by Anden and Mallat \citep{1dscattering1, ws}, which was later extended to 2 \citep{2dscattering} and 3 dimensions \citep{3dscattering, harmonicscattering}. Additional and more specialised forms for \ac{ws} include rotation-invariant scattering \citep{groupinvariantscattering} for the classification of textures and joint-\ac{tf} scattering \citep{ws_joint_tf, jointtfscattering2}. Joint-\ac{tf} scattering utilises a separable \ac{2d} filter that operates on the first level scattering scalogram, denoted by the operator $\mathcal{U}_1$, which is the only apparent usage of separable filters in the current scattering literature. 

In a deep-learning setting, sensible initialisation of filters prior to learning can significantly improve performance and interpretability \citep{sincnet}. The greatest advantage of separable filters is their computation speed when computing convolutions directly, as is performed in \acp{cnn} \citep{separablecnn}. Additionally, separable filters reduce the number of parameters of a \ac{cnn} if the filters are configured to be learnable. Learnable filters can typically improve performance compared to their fixed-filters counterpart \citep{scattering_birdsong}. 

At first glance, a \ac{ws} decomposition seems to be prohibitively expensive, but fast algorithms are possible due to the limited bandwidth of filter output. Fast algorithms utilise downsampling to take advantage of the demodulation of band-limited signals that result from the convolution of analytic wavelets \citep{2dscattering, 3dscattering, 1dscattering1}. Path pruning is also used to ignore filter combinations which have negligible energy. These algorithms are the standard implementation on many platforms, such as the Kymatio python package \citep{kymatio} and MATLAB \citep{MATLAB}.

The Morlet wavelet is the most widely used wavelet filter for a scattering filter bank implementation. Since the Morlet does not have compact support, implementations vary when considering the bandwidth/time support of the Morlet. This affects how Morlet \ac{fir} filters are discretised and truncated, how scattering paths with negligible energy are pruned, and how filters overlap in the frequency domain. 

In this work, we generalise \ac{1d} scattering to an arbitrary number of dimensions, which we refer to as the separable scattering transform. We propose flexible definitions for the Morlet bandwidth and filter overlap, which allows for an exact definition of near-optimal convolution computations with downsampling. Our implementation of separable scattering is \ac{gpu} accelerated, similar to the Kymatio implementations \citep{kymatio}. 

We demonstrate using the MNIST handwritten digit dataset \citep{mnist} that separable wavelet filters still perform adequately in a \ac{nn} classification setting compared to the conventional 2D scattering filters. We demonstrate the arbitrarily scalable dimensionality of the separable scattering transform with three-dimensional medical datasets from MedMNIST3D \citep{medmnist}. Separable scattering coefficients show \ac{sota} results for some of the MedMNIST3D datasets when combined with a simple \ac{nn}, while performing comparably to \ac{cnn} benchmarks on datasets which do not indicate \ac{sota} results.

\section{Separable Morlet Filterbank}

% In this section, we consider the continuous Morlet and provide general definitions of bandwidth and time-support for multiple dimensions. We utilise the Gaussian window to construct $n$-dimensional Morlets of varying characteristics.

\subsection{Morlet}

We define a \ac{1d} zero-mean Gaussian as
\begin{equation}
    \theta_{\sigma_t}(t) = \frac{1}{\sqrt{2\pi\sigma_t^2}}e^{-\frac{1}{2}\left(\frac{t}{\sigma_t}\right)^2},
\end{equation}
which has a Fourier transform transform $\theta_{\sigma_t}(t) \xleftrightarrow{\mathcal{F}} \hat{\theta}_{\sigma_\omega}(\omega)$, where $\sigma_\omega = \frac{1}{\sigma_t}$.

For an arbitrary bandwidth definition, we define the bandwidth-to-$\sigma$ ratio $\beta \in \mathbb{R}^+$, such that the one-sided bandwidth of a zero-mean Gaussian is $\beta \sigma_\omega$. For reference, Kymatio's implementation of 2D scattering indirectly defines $\beta \approx 2.5$ \citep{kymatio}.

% The $-3$ dB point of a Gaussian \ac{lpf} occurs at $\beta \approx 0.833$, but since the Gaussian \ac{lpf} does not have a very steep cutoff, values of $\beta \in [1, 3]$ may be appropriate to combat aliasing. 

A \ac{1d} Morlet $\uppsi$ has energy concentrated around 1 rad/s, with most of its energy contained in the interval $\omega \in [1 - \frac{1}{Q}, 1 + \frac{1}{Q}]$. $Q \in \mathbb{R}^+$ is defined as the number of wavelets per octave. The mother wavelet is given as
\begin{equation}
    \uppsi(t) = \theta_{\sigma_t}(t)\left(e^{jt} - \theta_0 \right),
\end{equation}
where $\theta_0 = \frac{\theta_{\sigma_t}(-1)}{\theta_{\sigma_t}(0)}$ to ensure zero mean: $\hat{\uppsi}(0) = 0$.

For reasons that will become apparent in section \ref{sec:filterbank}, we defined a 1D-wavelet dilated by a factor $\lambda$ as 
\begin{equation}
\label{eqn:dilwav}
    \uppsi_\lambda(t) = \begin{cases}
        \uppsi(\lambda t), \ \lambda \neq 0 \\
        \phi_j(t), \ \lambda = 0
    \end{cases},
\end{equation}
where $\phi$ represents the 1D \ac{lpf} utilised in scattering computations for the current dimension ($j$).

A $n$-dimensional separable wavelet may be constructed similarly, with
\begin{equation}
    \uppsi_{\vect{\lambda} }(\mathbf{u}) = \prod_{i=0}^{n-1} \uppsi_{\lambda_i}(u_i),
\end{equation}
where $\mathbf{u} = \left(u_1, ..., u_n\right)^T$ is the vector containing the dimensions of interest.

A filterbank is constructed by dilating the mother wavelet with a set of dilation factors $\vect{\lambda} =(\lambda_1, ..., \lambda_n )^T \in \Lambda_1 \times ... \times \Lambda_n$, with $\Lambda_i$ the set of dilation factors for the variable $u_i$. The dilation factor $\lambda_i$ is also the centre frequency of the dilated wavelet in rad/s. 

% Each dimension indexed by $i$ has its own defined \ac{lpf}.

% with with a dilated one-sided bandwidth of $\lambda_i \beta \sigma_{\omega_i}$. 

\subsection{1D Filterbank Construction}
\label{sec:filterbank}
 Suppose a 1D Morlet has a centre frequency $\lambda_0$. The following filter is placed at $\lambda_0 2^{\frac{1}{Q}}$. To define the amount of overlap between filters, it is useful express the corresponding filter's frequency \ac{std} $\sigma_\omega$ in terms of the distance between filters in the frequency domain. If the wavelet at $\lambda_0$ requires a decay equivalent to $\alpha \sigma_\omega$ \ac{std}s at the next wavelet at $\lambda_0 2^{\frac{1}{Q}}$. It follows that $\sigma_\omega = \frac{1}{\sigma_t} = \frac{1}{\alpha}\left( 2^\frac{1}{Q} - 1 \right)$.

We refer to $\alpha$ as the clearance factor. More overlap will result in more redundancy in the frequency representation, but also shorter filter impulse responses. 

% It is possible to set the clearance factor equal to the defined Gaussian bandwidth: $\alpha = \beta$. However, different values of $\alpha$ and $\beta$ allow for the separate definition of filter clearance and bandwidth, allowing the application to tune downsampling and filter time support separately. 

The provided definitions of filter clearance ($\alpha$) and bandwidth ($\beta$) may also be extended to non-separable wavelets, such as the rotational construction employed in \citep{2dscattering}. However, a conversion is required to find the bandwidth in each axis for a specific wavelet for rotationally constructed filterbanks, in order to employ the downsampling strategies proposed in this paper.

% A \ac{1d} wavelet filterbank is constructed with a set of filters $\mathbb{F} := \left\{\uppsi_\lambda \middle| \lambda \in \Lambda\right\}$ and a \ac{lpf} $\phi$ with the following specified parameters:
% \begin{enumerate}
%     \item $Q$ - wavelets per octave.
%     \item  $\omega_0$ - starting frequency in rad/s (optional).
%     \item  $d$ - invariance scale in samples, equivalent to the decimation factor applied to scattering coefficients.
% \end{enumerate}

All filterbank descriptions in this paper utilise normalised sampling frequency, i.e., the sample frequency is $f_s = 1$.

For some audio applications, for example speech recognition, low-frequency content is negligible, motivating filterbank construction starting at a specified frequency $\omega_0$. Otherwise, the entire frequency domain may be covered by setting $\omega_0$ to start at a position specified by $\phi$ and $\alpha$.

The \ac{lpf} $\phi$ is constant across all layers in a \ac{1d} scattering transform, and is chosen as $\phi(t) = \theta_{\sigma_{\phi, t}}(t)$, where $\sigma_{\phi, t} = \dfrac{d}{ \pi \beta}$.

The scattering transform requires that the time support of all filters do not exceed the time support of $\phi$, ensuring no filters contain time information exceeding the invariance scale $d$. $d$ is equivalently defined as the total downsample factor of the scattering tranform, and is unique for each dimension. Equivalently, the frequency \ac{std} of the filters may not exceed the \ac{std} of $\phi(\omega)$. For brevity, we denote the frequency \ac{std}s as $\sigma_{\lambda, \omega}$ and $\sigma_{\phi, \omega}$ for the Morlets and the \ac{lpf} respectively. 

 Since the maximum time support (minimum bandwidth) is a function of $d$, all dilated wavelets have their Gaussian envelopes restricted to a maximum time support of $\sigma_{\phi, t}$. This requires linearly spaced filters until the dilated bandwidth is larger than the \ac{lpf} bandwidth. To construct a set of positive dilation factors $\lambda \in \Lambda^+$, refer to algorithm \ref{alg:fb}.

\begin{algorithm}    
\caption{$\Lambda^+$ construction.}\label{alg:fb}
\begin{algorithmic}
    \State $\sigma_{\phi, \omega} \gets \dfrac{\pi\beta}{d}$
    \If{frequency limited}
        \Ensure{$\omega_0 \ge \dfrac{\pi\alpha}{d}$}
        \State $\lambda \gets \omega_0$
    \Else{}
        \State $\lambda \gets \dfrac{\pi\alpha}{d}$
    \EndIf
    \State $\Lambda^+ \gets \varnothing$
    \State $\sigma_\omega \gets \dfrac{1}{\alpha}\left( 2^\frac{1}{Q} - 1 \right)$
    \State $\sigma_{\lambda, \omega} \gets \lambda\sigma_\omega$
    
    \While{$\sigma_{\phi, \omega} > \sigma_{\lambda, \omega}$ and $\lambda < \pi$} 
        \State $\Lambda^+ \gets \Lambda^+ \cup \{\lambda\}$
        \State $\lambda \gets \lambda + \alpha \sigma_{\phi, \omega}$ 
        \State $\sigma_{\lambda, \omega} \gets \lambda\sigma_\omega$
    \EndWhile
    
    \While{$\lambda < \pi$}
        \State $\Lambda^+ \gets \Lambda^+ \cup \{\lambda\}$
        \State $\lambda \gets \lambda 2^{\frac{1}{Q}}$
        \State $\sigma_{\lambda, \omega} \gets \lambda\sigma_\omega$
    \EndWhile
\end{algorithmic}
\end{algorithm}

Only positive $\lambda$'s have been defined thus far, which provides inadequate coverage of the frequency domain in multiple dimensions. For real input signals, it is only necessary to cover half of one of the dimensions (only positive $\lambda$'s), whereas full coverage (both negative and positive $\lambda$'s) is required for additional dimensions. On-axis coverage is also required, in which each $\uppsi_\lambda$ must be multiplied with a Gaussian (zero-frequency wavelet), which the dilated wavelet definition in equation (\ref{eqn:dilwav}) defines as $\lambda=0$. A similar construction procedure is followed in \citep{jointtfscattering2}.

Given $m$ 1D filterbanks, with $m \ge 2$, each having Morlet filters with a positive set of lambdas $\Lambda_i^+$ and invariance scales $\vect{d} \in \mathbb{N}^m$, with $i = 1, ..., m$ indexing the dimension, we construct the $m$-dimensional filterbank with 
\begin{equation}
    \mathbb{F} = \left\{ \uppsi_{\vect{\lambda} }(\mathbf{u}) \ \middle| \ \vect{\lambda} \in \mathbb{L} \setminus \{\vect{0}\} \right\},
\end{equation}
where $\vect{u}$ is the $m$-dimensional spatial and/or time variable in which the each 1D filterbank is defined.  By definition, 
\begin{gather}
    \mathbb{L} = (\Lambda_1^+ \!\cup\! \{0\})\!\times\! (\Lambda_2^+ \!\cup\! \Lambda_2^- \!\cup \!\{0\}) \!\times\! ... \!\times\! (\Lambda_m^+ \!\cup\! \Lambda_m^- \!\cup\! \{0\}); \\
    \Lambda_i^- = \left\{-\lambda \ \middle| \ \lambda \in \Lambda_i^+\right\},
\end{gather}
where $\cup$ indicates the set union operator and $\times$ the Cartesian product. For $m=1$, the provided definitions result in a conventional \ac{1d} scattering transform \citep{1dscattering1}.

The $m$-dimensional \ac{lpf} is defined as 
\begin{equation}
    \phi(\vect{u}) = \uppsi_{\vect{0}}(\vect{u}).
\end{equation}

% The dilated wavelet definition provided by (\ref{eqn:dilwav}) therefore provides a concise set construction, with which to construct the \ac{lpf} and on-axis wavelets simultaneously.

% All on-axis filters will be highly correlated with their negative counterpart due to the symmetry of the multidimensional Fourier transform of real signals. For example, the 2D filter specified by $\lambda = [0, x]^T$ will be highly correlated with the filter at $[0, -x]^T$. As the invariance scale $d$ increases, the on-axis filters will become more narrow in the zero coordinate dimension, thereby becoming increasingly correlated.


% \subsection{Discretisation}

% Morlets do not have compact support, as such, they must be truncated for computation. The invariance scale $d$ specifies the maximum allowable time support of all filters. Since input signals must be at least of length $d$, filters can be precomputed using the padded signal length, while also guaranteeing a suitable level of truncation. The padding required is on the order of $d$ samples.

% Note that, due to their separability, all $\uppsi_{\vect{\lambda}} \in \mathbb{F}$ can be stored as $d$ one-dimensional filters, and not as $N_1 \times ... \times N_d$ tensors, thereby saving on the storage required for pre-computed filters. The savings on storage space allow implementations to compute the filters at multiple input sample frequencies, allowing for optimal downsampling strategies without incurring a significant storage cost.

\section{Separable Scattering Transform} \label{sec:wst}
% In this section, we describe the efficient computation of $m$-dimensional separable scattering transforms. The described downsampling strategy is utilised to a limited extent in Kymatio scattering implementations \citep{kymatio} and the original proposed fast algorithms \citep{2dscattering}, which is generalised further in this section.

\subsection{Transform}

The scattering transform requires 2 steps to provide scattering coefficients. The scalogram operator $\mathcal{U}_j$ iteratively filters a discrete signal $x[\vect{n}]$ for a given set of filters $\mathbb{F}$, which is then averaged by the \ac{lpf}. $\vect{n}$ represents a multidimensional index variable.
\begin{gather}
    \mathcal{U}_j x [\vect{n}, \vect{\lambda}_1, ..., \vect{\lambda}_j] = \left|\left(\mathcal{U}_{j-1} \ ...\  \mathcal{U}_1 x\right) * \uppsi_{\vect{\lambda}_j} \right|, \ \forall \ \uppsi_{\vect{\lambda}_j} \in \mathbb{F}; \\
     \mathcal{U}_1 x [\vect{n}, \vect{\lambda}_1] =  \left| x * \uppsi_{\vect{\lambda}_1} \right| , \ \uppsi_{\vect{\lambda}_1} \in \mathbb{F}.
\end{gather}

The scattering operator $\mathcal{S}_j$ provides the output coefficients at the $j$'th order of the scattering transform:
\begin{equation}
    \mathcal{S}_j x[\vect{n}, \vect{\lambda}_1, ..., \vect{\lambda}_j] = \mathcal{U}_j x * \phi.
\end{equation}
The \ac{lpf} $\phi$ remains constant throughout the transform. Note that the modulus/magnitude operator $|\cdot|$ demodulates the output of the filters, effectively extracting the Hilbert envelope from a band-limited signal \citep{waveletsandsubbandcoding}. 

The $j$'th scattering order adds an additional axis of paths indexed by $\vect{\lambda}_j$. However, not all paths need to be evaluated, since some paths have smaller bandwidths, thereby requiring fewer filters to extract the information lost by averaging. In particular, we only evaluate paths in which the centre frequencies of all elements of the vector $\vect{\lambda}_{j}$ are smaller than their corresponding bandwidth of the previous path's filter $\uppsi_{\vect{\lambda}_{j-1}}$. Path pruning is therefore dependent on $\alpha$ and $\beta$.

\subsection{Downsampling Strategy}
Since each filter specified by $\vect{\lambda}$ has its own bandwidth, we can employ downsampling across all paths non-uniformly. However, some care is required to ensure the compounded downsampling steps across all paths result in a uniform sampling frequency of the output scattering coefficients.

% The invariance scale $d$ is equivalent to the total downsampling required to compute scattering coefficients. $d$ is typically chosen as a power of $2$, which allows for multiple decimation stages to also subsample by powers of $2$. This decimation scheme is illustrated particularly well by dyadic wavelet transforms.

% However, it is not required that $d$ is restricted to a power of $2$, but rather that compounded decimation steps result in an effective decimation of $d$. As such, $d$ is most effective when it has as many factors as possible. In this study, we will require that $d$ is an even number.

Many applications are insensitive to small changes in $d$. As such, we propose a strategy to find an optimal $d$ given a target and tolerance value. For some applications, choosing $d$ such that the downsampling factor is a power of 2 is the simplest solution to achieve optimality. 

Without prior knowledge of the filterbank configurations, given a target invariance scale of $\bar{d}$ samples and a tolerance $\epsilon$, we can optimise $d \in \{\floor{(1 - \epsilon)\bar{d}}, \ceil{(1 + \epsilon)\bar{d}}\}$ such that it results in largest number of supported downsampling configurations.

A downsample factor $d$ which decomposes into a set of $n$ prime factors $\{p_1, ..., p_n\}$ with a corresponding multiset $\mathbb{M} = \left\{ m_1, ..., m_n \right\}$, where $m_i$ is the multiplicity of the prime $p_i$. We can find an optimal $d$ by maximising the sum $\sum\limits_{m \in M} m$.

% However, the proposed scheme for finding an optimal $d$ allows for more choices, thereby no restricting the choice of $d$ much, while also allowing for efficient dowmsampling.

Morlet filters in a 1D filterbank may be downsampled by a factor $d_{\psi_1}$ prior to low-pass filtering, and then downsampling again by a factor $d_{\phi_1}$ after low-pass filtering. As such, the compounded effect of downsampling restricts $d = d_{\psi_1} \cdot d_{\phi_1}$.

In the second order of scattering, the process is repeated with an additional pre-low-pass downsampling factor of $d_{\psi_2}$. The second level application of $\phi$ then downsamples by a factor $d_{\phi_2}$. To maintain a consistent output sampling frequency, it restricts $d = d_{\psi_2} \cdot d_{\psi_1} \cdot d_{\phi_2}$. 

Continuing the downsampling chain, the $i$'th level of downsampling requires $d = d_{\psi_i} \cdot ... \cdot d_{\psi_1} \cdot d_{\phi_i}$. The output of each operation of $\mathcal{U}_i$ and $\mathcal{S}_i$ must be downsampled as much as possible in order to make subsequent operations faster. To ensure that the application of all the downsampling steps are efficient, we require $d$ to have as many prime factors as possible, including factor multiplicity, so that a larger variety of downsampling combinations may be supported.



% \begin{algorithm}    
% \caption{Finding an optimal $d$.}\label{alg:optt}
% \begin{algorithmic}
%     \State $d_1 \gets \floor*{\frac{f_s \bar{T}(1 - \epsilon)}{2 \beta}}$\\
%     \State $d_2 \gets \ceil*{\frac{f_s \bar{T}(1 + \epsilon)}{2 \beta}}$\\
%     \State $k_\text{max} \gets 1$\\
%     \For{$d_j \in \{d_1, d_1 + 1, \ ... \ , d_2 - 1, d_2\}$}
%         \State $M \gets \text{PrimeMultiplicity}(d_j)$
%         \State  $k \gets \sum\limits_{m \in M} m$\\
%         \If{$k > k_\text{max}$}
%             \State $k_\text{max} \gets k$
%             \State $d \gets d_j$
%         \EndIf
%     \EndFor
% \end{algorithmic}
% \end{algorithm}


Consider a 1D wavelet filterbank and a single wavelet filter $\uppsi_1[n]$, applied to a discrete-time signal $x$. The operations required to compute the scattering coefficients is then notated for simplicity as
\begin{gather}
    u_1[n] = \left|  x * \uppsi_1   \right|; \\
    s_1[n] = y_1 * \phi.
\end{gather}

The bandwidth of $u_1$ is the bandwidth of an arbitrary first filter $\uppsi_1$. This follows from the Hilbert envelope computed by the analytic wavelet filter $\uppsi_\lambda$ and the modulus $|\cdot |$. The bandwidth of $s_1$ is the bandwidth of $\phi$.

Critical downsampling of a wavelet $\uppsi$ with a bandwidth of $\sigma_\omega$ is achieved by a factor of $d_{\uppsi} = \floor*{\frac{ \pi }{\beta\sigma_\omega}}$. Critical downsampling of $s_1$, is achieved via a factor of $d$, by definition.

We can efficiently compute $s_1$ using compounded downsampling steps:
\begin{equation}
    \left(s_1\right)_{\downarrow d} = \left( \left(u_1\right)_{\downarrow d_1} * (\phi)_{\downarrow d_1}\right)_{\downarrow d_2},
\end{equation}
such that $d = d_1 \cdot d_2, \ d_1, d_2 \in \mathbb{N}^+$, with $d_1 | d$ and $d_1 \le  d_{\uppsi_1}$. In order to find $d_1$, we decrement $d_{\uppsi_1}$ until it divides $d$ evenly. Each scalogram $u_j[n]$ is not necessarily downsampled optimally, but has a downsampling factor which guarentees a consistent scattering coefficient output sample frequency. 

% This strategy may counter intuitively result in faster computations for filterbanks with more filters, since a higher $Q$ implies a smaller bandwidth of each filter. 

A second order of scattering with a filter path of $(\uppsi_1, \uppsi_2)$ is performed on the downsampled $u_1$:
\begin{gather}
    u_2 = \left|  \left(u_1\right)_{\downarrow d_1} * (\uppsi_2)_{\downarrow d_1}  \right|; \\
    \left(s_2\right)_{\downarrow d} = \left( \left(u_2\right)_{\downarrow d_3} * (\phi)_{\downarrow d_3}\right)_{\downarrow d_4},
\end{gather}
such that $d = d_1 \cdot d_3 \cdot d_4, \ d_1, d_3, d_4 \in \mathbb{N}^+, d_3 \le d_{\uppsi_2}, d_3 | \frac{d}{d_1}$. The proposed downsampling scheme can be extended to an arbitrary number of scattering orders.

% Since the filterbank configuration is known, all required combinations of downsampled filters and their downsampling factors can be precomputed.




\subsection{Convolutions}
Optimal \ac{fft} convolutions can be achieved by performing downsampling in the frequency domain instead of the time domain. It is straightforward to verify that $|x * \uppsi|_{\downarrow r} = \left|(x * \uppsi)_{\downarrow r}\right|$, since the modulus is an element-wise operation, for some downsampling factor $r$. Given the signal and filter Fourier transforms $\hat{x}$ and $\hat{\uppsi}$, we then have
\begin{equation}
\label{eqn:circconv}
    x \otimes \uppsi [n] \xleftrightarrow{\mathcal{FFT}}  \hat{x} \cdot \hat{\uppsi} [k],
\end{equation}
where the $\otimes$ operator represents a circular convolution and $k$ is the frequency index.

Given that $r|N$, we can express (\ref{eqn:circconv}) when downsampled as a periodised summation \citep{waveletsandsubbandcoding} in the frequency domain
\begin{equation}
    (x \otimes \uppsi [n])_{\downarrow r} \xleftrightarrow{\mathcal{FFT}}  \frac{1}{r} \sum_{i=0}^{r-1} \hat{x} \cdot \hat{\uppsi} [k + iN/r], %https://citeseerx.ist.psu.edu/document?repid=rep1&type=pdf&doi=5216ea733e562541b33a7f97dab0de072b2e8827
\end{equation}
which can be efficiently implemented via shape manipulation of tensors in computational packages like MATLAB or PyTorch \citep{pytorch}. To compute valid convolutions, we must pad $x$ and $\uppsi$ to have a total length of $N = N_x + d + c$, where $c \in \mathbb{N}^+$ is a constant that ensures that $d | N$. 

% Note that only $d$ additional samples are required instead of $2d$, since $d$ is, by definition, the time support of $\phi$ in number of samples. Padding is applied equally to the left and right of $x$. For further reduction of boundary effects, reflection padding is utilised. Kymatio's 2D scattering implementation utilises a similar padding scheme. 

% Thus far, the techniques discussed may be equivalently implemented on many scattering transforms. Separable scattering benefits from reduced computations resulting from filter separability. In particular, given a $m$-dimensional signal $x[\vect{n}]$ indexed by $\vect{n} = (n_1, ..., n_m)^T$, with each dimension having downsample factors expressed as a vector $\vect{r} = (r_1, ..., r_m)^T$, we can chain downsampling for each dimension:
% \begin{equation}
%     \left(x * \uppsi_{\vect{\lambda}}[\vect{n}]\right )_{\downarrow \vect{r}} = \left( \left(x[\vect{\lambda}] * \uppsi_{\lambda_i}[n_i]\right)_{\downarrow r_1} ... * \uppsi_{\lambda_m}[n_m]\right)_{\downarrow r_m}.
% \end{equation}.

% We abuse notation to indicate that $\downarrow r_i$ is an operator that downsamples the $i$'th dimension, and $\downarrow \vect{r}$ downsamples all dimensions each with their correponding downsample factor in $\vect{r} \in \mathbb{N}^m$.

% For $m$-dimensional sequences of dimensional lengths $N_1, ..., N_m$, the forward and inverse \acp{fft} computation time is reduced from $2\sum_{i} T \log N_i$ (no frequency periodisation), with $T = \prod_i N_i$ to $\sum_{i} \frac{T}{K_i} \log \frac{N_i}{K_i} + P \log P$, where $K_1 = 1$, $K_i = \prod_{j < i} r_j$ and $P = \prod_i \frac{N_i}{r_i}$. However, the number of convolution multiplications in the frequency domain increases from $\prod_i N_i$ to $\sum_i \prod_j \frac{N_j}{K_i} \leq m \prod_i N_i$.

% In contrast, optimal non-separable scattering transforms can achieve a computational cost of $\sum_{i} N_i \log N_i + P \log P$, since a full forward \ac{fft} is required to compute a convolution with a non-separable $m$-dimensional filter.

% The greatest computational advantage is during direct convolution computations, as is present in \acp{cnn}. For an $m$-dimensional filter impulse response with filter lengths $L_1, ..., L_m$, the computational cost reduces from $P\prod_i L_i$ to $P \sum_i L_i$.


\section{Results}



\subsection{MNIST}

% We benchmark separable \ac{ws} against 2D scattering by repeating the MNIST hand-written digit dataset experiment in \citep{2dscattering}. 

The MNIST dataset \citep{mnist} contains 60000 training and 10000 test samples. Unlike in \citep{2dscattering}, which decorrelates scattering coefficients with a discrete cosine transform prior to classification, we perform classification on the scattering coefficients directly. Kymatio \citep{kymatio} is used to produce the 2D scattering coefficients. Our implementation of separable scattering is implemented similarly to Kymatio, with PyTorch \citep{pytorch} as a backend for FFT convolutions and \ac{nn} models.

Unless specified otherwise, all experiments have $\beta = \alpha = 2.5$. Scattering features are normalised prior to classification, according to the mean and \ac{std} calculated on the training set. No data augmentation is performed.

\begin{table}
\centering
\caption{MNIST classification error rate (\%) of separable and \ac{2d} scattering coefficients using a \ac{nn} classifier} \label{tab:mnistnn}
\begin{tabular}{|lr|l|} \hline
\multirow{2}{*}{2D WS + NN}        & $l=1$, $J=2$  & $0.64 \pm 0.05$    \\
          & $l=2$, $J=3$ & $0.50 \pm 0.03$     \\ \hline
\multirow{2}{*}{Separable WS + NN}  & $l=1$, $d=(4,4)$ & $0.63 \pm 0.05 $    \\
          & $l=2$, $d=(4,4)$  & $0.52 \pm 0.04$     \\ \hline
\end{tabular}
\end{table}


% Table \ref{tab:mnistlda} shows the results for MNIST using a \ac{lda} classifier from the Scikit-learn python package \citep{sklearn}, with a covariance shrinkage factor of $5\cdot 10^{-3}$. Scattering is performed for $l=1, 2$ levels.

% \begin{table}[!h] 
% \centering
% \caption{MNIST classification error rate (\%) of separable and \ac{2d} scattering coefficients  using an \ac{lda} classifier} \label{tab:mnistlda}
% \begin{tabular}{|r|rr|rr|} \hline
% \multirow{2}{*}{Training size}              & \multicolumn{2}{l}{2D}                               & \multicolumn{2}{l|}{Separable}                             \\
%  & \multicolumn{1}{l}{$l=1$} & \multicolumn{1}{l}{$l=2$} & \multicolumn{1}{l}{$l=1$} & \multicolumn{1}{l|}{$l=2$} \\ \hline
% 1000                              & 4.58                        & 2.69                        & 4.87                        & 4.58                        \\
% 2000                              & 3.89                        & 1.62                        & 4.31                        & 2.6                         \\
% 5000                              & 3.54                        & 1.12                        & 3.84                        & 1.7                         \\
% 10000                             & 3.35                        & 1.05                        & 3.54                        & 1.5                         \\
% 20000                             & 3.16                        & 0.87                        & 3.41                        & 1.41                        \\
% 40000                             & 2.92                        & 0.86                        & 3.34                        & 1.25                        \\
% 60000                             & 2.92                        & 0.81                        & 3.27                        & 1.34     \\ \hline                  
% \end{tabular}
% \end{table}

Due to its separability and non-angularly spaced filters, separable scattering does not perform as well compared to 2D scattering when using simple classifiers, such as \ac{lda} \citep{lda}. To illustrate that this performance discrepancy is not of significant consequence in a \ac{nn} setting, we test performance on the full dataset utilising a simple architecture. The neural network architecture used has an input layer with 256 neurons, followed by two hidden layers with 128 and 64 neurons respectively. The output layer has 10 neurons - one for each digit. Input and hidden layers are followed by a batch norm layer \citep{batchnorm} and ReLU activation function \citep{relu}. The output layer is followed by a softmax function. The Adam optimiser \citep{adam} is used with cross-entropy loss, a batch size of 256 and learning rate of $3 \cdot 10^{-5}$. 5000 of the 60000 training samples are reserved for validation and removed from the training set. Training is stopped when validation loss starts to increase. The \ac{nn} model is initialised with random weights, and the experiment is repeated 50 times. Different invariance scales were tested, and the best results are reported in table \ref{tab:mnistnn}. Tests are repeated for $l \in \{1, 2\}$ levels of scattering.





\subsection{MedMNIST3D}


The MedMNIST3D datasets are a subset of the MedMNIST dataset group \citep{medmnist}, where each 3D dataset contains $28\times 28 \times 28$ images with 2, 3 or 11 classes. Train, test and validation data partitions are provided by the authors. All datasets have on the order of 1000 training samples. We compare the baseline \ac{nn} results provided in \citep{medmnist} with separable scattering features combined with a simple \ac{nn} classifier.

We use \ac{nn} classifier with an input layer containing 1024 neurons, followed by two hidden layers with 512 and 256 neurons respectively. Input and hidden layers layers are each followed by a batch norm layer \citep{batchnorm} and a ReLU non-linearity \citep{relu}. For datasets with two classes, the output layer is a single neuron followed by a sigmoid activation function. For datasets with more than two classes, the output layer has a size equal to the number of classes, followed by a softmax activation. Binary cross-entropy loss are used for datasets with two classes, otherwise cross-entropy loss is used. The Adam optimiser \citep{adam} with a learning rate of $1 \cdot 10^{-5}$ is used. All other configuration parameters are identical to the model used for the MNIST dataset.

A single level of scattering coefficients are computed, with $Q = (2, 2, 2)$ and $d = (4, 4, 4)$. Many MedMNIST3D datasets tend to be unbalanced, implying that \ac{auc} is a more reliable metric to measure model performance. The results are shown in table \ref{tab:medmnist}, in which \ac{sota} \ac{auc} performance is achieved for the Organ, Adrenal and Vessel datasets. Table \ref{tab:medmnist} indicates the number of classes ($c$) for each of the datasets, with accuracy (ACC) also shown for reference. The performance of our method on non-\ac{sota} results are comparable to the other baseline \ac{nn} approaches presented in \citep{medmnist}. It is likely that better results can be achieved by the proposed method if the filters are made learnable and/or scattering parameters are uniquely optimised for each dataset.

\begin{table*}[t!]
    \centering
    \caption{MedMNIST classification results of compared to baseline \ac{nn} approaches (\citep{medmnist})} \label{tab:medmnist}
    \begin{adjustbox}{angle=90}
\begin{tabular}{|l|rr|rr|rr|rr|rr|rr|}
\hline
\multirow{2}{*}{Methods}                & \multicolumn{2}{c}{Organ ($c=11$)}                  & \multicolumn{2}{c}{Nodule ($c=2$)}                 & \multicolumn{2}{c}{Fracture ($c=3$)}               & \multicolumn{2}{c}{Adrenal ($c=2$)}                & \multicolumn{2}{c}{Vessel ($c=2$)}                 & \multicolumn{2}{c|}{Synapse ($c=2$)}                \\
                         & \multicolumn{1}{l}{AUC} & \multicolumn{1}{l}{ACC} & \multicolumn{1}{l}{AUC} & \multicolumn{1}{l}{ACC} & \multicolumn{1}{l}{AUC} & \multicolumn{1}{l}{ACC} & \multicolumn{1}{l}{AUC} & \multicolumn{1}{l}{ACC} & \multicolumn{1}{l}{AUC} & \multicolumn{1}{l}{ACC} & \multicolumn{1}{l}{AUC} & \multicolumn{1}{l|}{ACC} \\ \hline
ResNet-1810+2.5D        & 0.977                   & 0.788                   & 0.838                   & 0.835                   & 0.587                   & 0.451                   & 0.718                   & 0.772                   & 0.748                   & 0.846                   & 0.634                   & 0.696                   \\
ResNet-1810+3D          & 0.996                   & 0.907                   & 0.863                   & 0.844                   & 0.712                   & 0.508                   & 0.827                   & 0.721                   & 0.874                   & 0.877                   & 0.82                    & 0.745                   \\
ResNet-1810+ACS41       & 0.994                   & 0.900                     & 0.873                   & 0.847                   & 0.714                   & 0.497                   & 0.839                   & 0.754                   & 0.930                    & 0.928                   & 0.705                   & 0.722                   \\
ResNet-5010+2.5D        & 0.974                   & 0.769                   & 0.835                   & 0.848                   & 0.552                   & 0.397                   & 0.732                   & 0.763                   & 0.751                   & 0.877                   & 0.669                   & 0.735                   \\
ResNet-5010+3D          & 0.994                   & 0.883                   & 0.875                   & 0.847                   & 0.725                   & 0.494                   & 0.828                   & 0.745                   & 0.907                   & 0.918                   & \textbf{0.851}          & \textbf{0.795}          \\
ResNet-5010+ACS41       & 0.994                   & 0.889                   & 0.886                   & 0.841                   & \textbf{0.750}           & \textbf{0.517}          & 0.828                   & 0.758                   & 0.912                   & 0.858                   & 0.719                   & 0.709                   \\
auto-sklearn11          & 0.977                   & 0.814                   & \textbf{0.914}          & \textbf{0.874}          & 0.628                   & 0.453                   & 0.828                   & \textbf{0.802}          & 0.910                   & \textbf{0.915}          & 0.631                   & 0.730                    \\
AutoKeras12             & 0.979                   & 0.804                   & 0.844                   & 0.834                   & 0.642                   & 0.458                   & 0.804                   & 0.705                   & 0.773                   & 0.894                   & 0.538                   & 0.724                   \\ \hline
Separable WS + NN (Ours)  & \textbf{0.998}          & \textbf{0.941}          & 0.858                   & 0.797                   & 0.614                   & 0.458                   & \textbf{0.875}          & 0.792                   & \textbf{0.962}          & 0.895                   & 0.715                   & 0.525    \\ \hline              
\end{tabular}
\end{adjustbox}
\end{table*}



\acrodef{dct}[DCT]{discrete cosine transform}

\chapter{Feature Extraction with Wavelet Transforms}
\label{chap:p2i}

\ac{tf} decompositions are often used to extract features from signals. In audio, \acp{mfcc} are features calculated from the \ac{stft} of the signal. Note that a ``\ac{tf}'' decomposition, as referred to in this dissertation, does not necesarrily require a time variable: spatial variables are often present instead. In this sense, a \ac{cnn} is can also be considered as a ``\ac{tf}'' decomposition, where multi-dimensional index variables (or index vectors) are used for ``time'' and frequency:
\begin{equation}
    X[\vect{k}, \vect{m}] = \left( x * \upsilon_{\vect{k}} \right)_{\downarrow \vect{d}}, \ \upsilon_{\vect{k}} \in \Upsilon.
\end{equation}

In this case, we abuse notation to indicate that the operator $(\cdot)_{\downarrow \vect{d}}$ downsamples by different amounts for each dimension of $\vect{m}$. 

The downsampling factor $\vect{d}$ limits the bandwidth of $X$, which, as discussed in chapter \ref{chap:p1i}, is selected so as to retain all information carried by the filters in $\Upsilon$. Although not notated, different filters (different $\vect{k}$) may have different values of $\vect{d}$. Notation to support such a case is neglected for clarity, but is later expanded upon in chapter \ref{chap:p3}.

\section{The Importance of Invariance}

The decomposition $X$ is often further processed by additional filters $\Phi_i[\vect{k}, \vect{m}]$, where $i$ is the index of the post-processing filter:
\begin{equation}
    \tilde{X_i}[\vect{k}, \vect{m}] = X * \Phi_i.
\end{equation}
The filters $\Phi_i$ are not necesarrily multi-dimensional, and may only operate along a single dimension of $X$. If $\Phi_i$ is an averaging (low-pass) filter, some additional stability can be observed in $\tilde{X_i}$ compared with $X$. This allows for the introduction of invariances to various deformations applied to $X$, which can be greatly beneficial for some applications \citep{cnninvariance}. In a sense, cascased structures of $X$ and $\tilde{X_i}$ are defined in a \ac{cnn} for which the network learns the required invariances \cite{2dscattering}.

It is widely known that enforcing invariance in feature extraction and \ac{ml} pipelines increase performance. Enforcing invariance in \acp{nn} has been shown to improve the network, while also improving the interpretability of the network \cite{cnninvariance2}. For example, when classifying images, invariance to scale, shear, rotation and translation deformations, de\-pending on image properties, may be useful when extracting features. Specifically, defor\-mations of a specified order should approximately map to the same point in feature space:
\begin{equation}
    \Gamma\left(\delta\left(x\left[\vect{n}\right]\right)\right) \approx \Gamma\left(x\left[\vect{n}\right]\right),
\end{equation}
where $\Gamma$ is the feature extraction operator, and $\delta$ is the deformation operator applied to $x$.

Invariance filters ($\Phi_i$) can also have additional properties that makes it favourable for audio applications. For example, many applications ``blur'' (low-pass) the TF decomposition \citep{seyicwt, speechmodulation}, which can reduce the impact of noise fluctuations, and can also serve to suppress small frequency fluctuations across adjascent frequency bins (if the blurring kernel $\Phi_i$ filters across frequency as well). The blurring operation can improve the performance of signal detectors and classifiers. However, as a cost, time localisation and/or frequency localisation is reduced.

A new problem is introduced when enforcing invariances via the $\Phi_i$ filters - information is lost via averaging \cite{1dscattering1}. Various techniques can be used to further retain information, while also retaining invariance. Wavelet scattering is such a method \cite{ws,2dscattering}. In a sense, multilayer \acp{cnn} topologies with skip connections can be thought of as a method to recover lost information in the deeper layers.

\section{MFCCs Reframed as Shift-Invariant Features}

\acp{mfcc} are the most widely used feature extraction method for audio signals \citep{mfccreview}, save for \ac{cnn} frontends which also act as a feature extractor when operating on the \ac{stft}. For this reframing, we ignore the \ac{dct} applied to the filtered values, since this linear transformation only serves to decorrelate the filter coefficients \citep{dctdecorrelation}.

The magnitude spectrum used in \ac{mfcc} calculation is expressed as a \ac{tf} decomposition as defined by equations (\ref*{eqn:gentf}) and (\ref*{eqn:stftfilter}):
\begin{equation}
    X[l, m] = |\upsilon_l * x|_{\downarrow d},
\end{equation}
where $\upsilon_l$ is the $l$'th \ac{stft} filter.

The Mel spectrogram \ac{tf} decomposition $X_\text{Mel}$ is then calculated as
\begin{equation}
    X_\text{Mel}[k, m] = \log\left(X[l, m] * \Phi_k[l]\right) \bigg|_{l = 0},
\end{equation}
where $\Phi_k$ is the $k$'th MFFC triangular filter operating along the STFT index $l$. The evaluation at $l=0$ indicates that this operation is a multiplication and summation only. 

We can view this process as a \ac{tf} decomposition with STFT filters $\upsilon_l \in \Upsilon$, which is modified with the Mel-scale triangular filters $\Phi_k$. $\upsilon_l$ provides time-shift invariance due to the STFT window, whereas $\Phi_k$ selectively averages frequency content of the STFT, thereby providing some frequency-shift invariance.


\section{Wavelets as a MFCC Generalisation}

We can construct a wavelet filterbank which introduces time and frequency-shift invariance in a similar manner to the Mel spectrogram. This process is constructed in the opposite order (frequency-shift, then time-shift invariance) to the Mel spectrogram. However, its properties remain similar \cite{ws}.

We construct wavelet band-pass filters $\upsilon_k \in \Upsilon$ which is spaced in frequency such that the desired invariance properties is obtained (as discussed in section \ref*{sec:p1i:wavelets}). We then utilise a single low-pass filter $\Phi[\vect{m}]$ which then provides invariance in the time and/or spatial dimensions:
\begin{gather}
    \label{eqn:general_scalogram}
    X[\vect{k}, \vect{m}] = \left|x * \upsilon_{\vect{k}}\right|_{\downarrow \vect{d}}, \ \upsilon_{\vect{k}} \in \Upsilon; \\
    \label{eqn:general_scattering}
    \tilde{X}[\vect{k}, \vect{m}] = X[\vect{k}, \vect{m}] * \Phi[\vect{m}]
\end{gather}

The notation in equations (\ref*{eqn:general_scalogram}) and (\ref*{eqn:general_scattering}) conforms the general \ac{tf} description of chapter \ref*{chap:p1i}, although the specific wavelet scattering operators in chapters \ref*{chap:p2} to \ref*{chap:p3} follow standard notation \citep{waveletbook}. These operators are known as wavelet scattering, which formalise the notion of ``averaging a \ac{tf}-decomposition'' in terms of signal processing theory. Chapters \ref*{chap:p2} to \ref*{chap:p3} discuss the specifics, flavours and implementation of scattering operators at length.


\chapter{Separable Wavelet Scattering}
\label{chap:p3}




\articleinfo{M.W. Rademan, D.J.J Versfeld, J.A. Du Preez}{IEEE Signal Processing Letters}{15 May 2024}{doi}{Wavelet scattering is a widely used feature extraction method due to it efficacy in extracting invariant features, 
while retaining any lost high frequency information resulting from averaging to obtain the desired amount of invariance. 
By generalising a 1-dimensional scattering transform, we extend its definition to an arbitrary number of independent dimensions. 
We show that, in a modern deep learning setting, separable wavelet scattering performs similarly to its non-separable counterparts 
with the MNIST hand-written digit dataset. We additionally demonstrate state-of-the-art results for a subset of the MedMNIST3D datasets.}


\section{Introduction}
\Ac{ws} has seen widespread use in classification applications as a powerful feature extraction method. It is an especially effective method for small datasets, since the feature filters are reminiscent of a \ac{cnn} front-end, while requiring no learning. The resulting features are invariant across all averaged dimensions, and exhibit separable class subspaces, allowing linear classifiers to be used with high efficacy \citep{2dscattering}.

Various forms of \ac{ws} exist, with \ac{1d} scattering first proposed by Anden and Mallat \citep{1dscattering1, ws}, which was later extended to 2 \citep{2dscattering} and 3 dimensions \citep{3dscattering, harmonicscattering}. Additional and more specialised forms for \ac{ws} include rotation-invariant scattering \citep{groupinvariantscattering} for the classification of textures and joint-\ac{tf} scattering \citep{ws_joint_tf, jointtfscattering2}. Joint-\ac{tf} scattering utilises a separable \ac{2d} filter that operates on the first level scattering scalogram, denoted by the operator $\mathcal{U}_1$, which is the only apparent usage of separable filters in the current scattering literature. 

In a deep-learning setting, sensible initialisation of filters prior to learning can significantly improve performance and interpretability \citep{sincnet}. The greatest advantage of separable filters is their computation speed when computing convolutions directly, as is performed in \acp{cnn} \citep{separablecnn}. Additionally, separable filters reduce the number of parameters of a \ac{cnn} if the filters are configured to be learnable. Learnable filters can typically improve performance compared to their fixed-filters counterpart \citep{scattering_birdsong}. 

At first glance, a \ac{ws} decomposition seems to be prohibitively expensive, but fast algorithms are possible due to the limited bandwidth of filter output. Fast algorithms utilise downsampling to take advantage of the demodulation of band-limited signals that result from the convolution of analytic wavelets \citep{2dscattering, 3dscattering, 1dscattering1}. Path pruning is also used to ignore filter combinations which have negligible energy. These algorithms are the standard implementation on many platforms, such as the Kymatio python package \citep{kymatio} and MATLAB \citep{MATLAB}.

The Morlet wavelet is the most widely used wavelet filter for a scattering filter bank implementation. Since the Morlet does not have compact support, implementations vary when considering the bandwidth/time support of the Morlet. This affects how Morlet \ac{fir} filters are discretised and truncated, how scattering paths with negligible energy are pruned, and how filters overlap in the frequency domain. 

In this work, we generalise \ac{1d} scattering to an arbitrary number of dimensions, which we refer to as the separable scattering transform. We propose flexible definitions for the Morlet bandwidth and filter overlap, which allows for an exact definition of near-optimal convolution computations with downsampling. Our implementation of separable scattering is \ac{gpu} accelerated, similar to the Kymatio implementations \citep{kymatio}. 

We demonstrate using the MNIST handwritten digit dataset \citep{mnist} that separable wavelet filters still perform adequately in a \ac{nn} classification setting compared to the conventional 2D scattering filters. We demonstrate the arbitrarily scalable dimensionality of the separable scattering transform with three-dimensional medical datasets from MedMNIST3D \citep{medmnist}. Separable scattering coefficients show \ac{sota} results for some of the MedMNIST3D datasets when combined with a simple \ac{nn}, while performing comparably to \ac{cnn} benchmarks on datasets which do not indicate \ac{sota} results.

\section{Separable Morlet Filterbank}

% In this section, we consider the continuous Morlet and provide general definitions of bandwidth and time-support for multiple dimensions. We utilise the Gaussian window to construct $n$-dimensional Morlets of varying characteristics.

\subsection{Morlet}

We define a \ac{1d} zero-mean Gaussian as
\begin{equation}
    \theta_{\sigma_t}(t) = \frac{1}{\sqrt{2\pi\sigma_t^2}}e^{-\frac{1}{2}\left(\frac{t}{\sigma_t}\right)^2},
\end{equation}
which has a Fourier transform transform $\theta_{\sigma_t}(t) \xleftrightarrow{\mathcal{F}} \hat{\theta}_{\sigma_\omega}(\omega)$, where $\sigma_\omega = \frac{1}{\sigma_t}$.

For an arbitrary bandwidth definition, we define the bandwidth-to-$\sigma$ ratio $\beta \in \mathbb{R}^+$, such that the one-sided bandwidth of a zero-mean Gaussian is $\beta \sigma_\omega$. For reference, Kymatio's implementation of 2D scattering indirectly defines $\beta \approx 2.5$ \citep{kymatio}.

% The $-3$ dB point of a Gaussian \ac{lpf} occurs at $\beta \approx 0.833$, but since the Gaussian \ac{lpf} does not have a very steep cutoff, values of $\beta \in [1, 3]$ may be appropriate to combat aliasing. 

A \ac{1d} Morlet $\uppsi$ has energy concentrated around 1 rad/s, with most of its energy contained in the interval $\omega \in [1 - \frac{1}{Q}, 1 + \frac{1}{Q}]$. $Q \in \mathbb{R}^+$ is defined as the number of wavelets per octave. The mother wavelet is given as
\begin{equation}
    \uppsi(t) = \theta_{\sigma_t}(t)\left(e^{jt} - \theta_0 \right),
\end{equation}
where $\theta_0 = \frac{\theta_{\sigma_t}(-1)}{\theta_{\sigma_t}(0)}$ to ensure zero mean: $\hat{\uppsi}(0) = 0$.

For reasons that will become apparent in section \ref{sec:filterbank}, we defined a 1D-wavelet dilated by a factor $\lambda$ as 
\begin{equation}
\label{eqn:dilwav}
    \uppsi_\lambda(t) = \begin{cases}
        \uppsi(\lambda t), \ \lambda \neq 0 \\
        \phi_j(t), \ \lambda = 0
    \end{cases},
\end{equation}
where $\phi$ represents the 1D \ac{lpf} utilised in scattering computations for the current dimension ($j$).

A $n$-dimensional separable wavelet may be constructed similarly, with
\begin{equation}
    \uppsi_{\vect{\lambda} }(\mathbf{u}) = \prod_{i=0}^{n-1} \uppsi_{\lambda_i}(u_i),
\end{equation}
where $\mathbf{u} = \left(u_1, ..., u_n\right)^T$ is the vector containing the dimensions of interest.

A filterbank is constructed by dilating the mother wavelet with a set of dilation factors $\vect{\lambda} =(\lambda_1, ..., \lambda_n )^T \in \Lambda_1 \times ... \times \Lambda_n$, with $\Lambda_i$ the set of dilation factors for the variable $u_i$. The dilation factor $\lambda_i$ is also the centre frequency of the dilated wavelet in rad/s. 

% Each dimension indexed by $i$ has its own defined \ac{lpf}.

% with with a dilated one-sided bandwidth of $\lambda_i \beta \sigma_{\omega_i}$. 

\subsection{1D Filterbank Construction}
\label{sec:filterbank}
 Suppose a 1D Morlet has a centre frequency $\lambda_0$. The following filter is placed at $\lambda_0 2^{\frac{1}{Q}}$. To define the amount of overlap between filters, it is useful express the corresponding filter's frequency \ac{std} $\sigma_\omega$ in terms of the distance between filters in the frequency domain. If the wavelet at $\lambda_0$ requires a decay equivalent to $\alpha \sigma_\omega$ \ac{std}s at the next wavelet at $\lambda_0 2^{\frac{1}{Q}}$. It follows that $\sigma_\omega = \frac{1}{\sigma_t} = \frac{1}{\alpha}\left( 2^\frac{1}{Q} - 1 \right)$.

We refer to $\alpha$ as the clearance factor. More overlap will result in more redundancy in the frequency representation, but also shorter filter impulse responses. 

% It is possible to set the clearance factor equal to the defined Gaussian bandwidth: $\alpha = \beta$. However, different values of $\alpha$ and $\beta$ allow for the separate definition of filter clearance and bandwidth, allowing the application to tune downsampling and filter time support separately. 

The provided definitions of filter clearance ($\alpha$) and bandwidth ($\beta$) may also be extended to non-separable wavelets, such as the rotational construction employed in \citep{2dscattering}. However, a conversion is required to find the bandwidth in each axis for a specific wavelet for rotationally constructed filterbanks, in order to employ the downsampling strategies proposed in this paper.

% A \ac{1d} wavelet filterbank is constructed with a set of filters $\mathbb{F} := \left\{\uppsi_\lambda \middle| \lambda \in \Lambda\right\}$ and a \ac{lpf} $\phi$ with the following specified parameters:
% \begin{enumerate}
%     \item $Q$ - wavelets per octave.
%     \item  $\omega_0$ - starting frequency in rad/s (optional).
%     \item  $d$ - invariance scale in samples, equivalent to the decimation factor applied to scattering coefficients.
% \end{enumerate}

All filterbank descriptions in this paper utilise normalised sampling frequency, i.e., the sample frequency is $f_s = 1$.

For some audio applications, for example speech recognition, low-frequency content is negligible, motivating filterbank construction starting at a specified frequency $\omega_0$. Otherwise, the entire frequency domain may be covered by setting $\omega_0$ to start at a position specified by $\phi$ and $\alpha$.

The \ac{lpf} $\phi$ is constant across all layers in a \ac{1d} scattering transform, and is chosen as $\phi(t) = \theta_{\sigma_{\phi, t}}(t)$, where $\sigma_{\phi, t} = \dfrac{d}{ \pi \beta}$.

The scattering transform requires that the time support of all filters do not exceed the time support of $\phi$, ensuring no filters contain time information exceeding the invariance scale $d$. $d$ is equivalently defined as the total downsample factor of the scattering tranform, and is unique for each dimension. Equivalently, the frequency \ac{std} of the filters may not exceed the \ac{std} of $\phi(\omega)$. For brevity, we denote the frequency \ac{std}s as $\sigma_{\lambda, \omega}$ and $\sigma_{\phi, \omega}$ for the Morlets and the \ac{lpf} respectively. 

 Since the maximum time support (minimum bandwidth) is a function of $d$, all dilated wavelets have their Gaussian envelopes restricted to a maximum time support of $\sigma_{\phi, t}$. This requires linearly spaced filters until the dilated bandwidth is larger than the \ac{lpf} bandwidth. To construct a set of positive dilation factors $\lambda \in \Lambda^+$, refer to algorithm \ref{alg:fb}.

\begin{algorithm}    
\caption{$\Lambda^+$ construction.}\label{alg:fb}
\begin{algorithmic}
    \State $\sigma_{\phi, \omega} \gets \dfrac{\pi\beta}{d}$
    \If{frequency limited}
        \Ensure{$\omega_0 \ge \dfrac{\pi\alpha}{d}$}
        \State $\lambda \gets \omega_0$
    \Else{}
        \State $\lambda \gets \dfrac{\pi\alpha}{d}$
    \EndIf
    \State $\Lambda^+ \gets \varnothing$
    \State $\sigma_\omega \gets \dfrac{1}{\alpha}\left( 2^\frac{1}{Q} - 1 \right)$
    \State $\sigma_{\lambda, \omega} \gets \lambda\sigma_\omega$
    
    \While{$\sigma_{\phi, \omega} > \sigma_{\lambda, \omega}$ and $\lambda < \pi$} 
        \State $\Lambda^+ \gets \Lambda^+ \cup \{\lambda\}$
        \State $\lambda \gets \lambda + \alpha \sigma_{\phi, \omega}$ 
        \State $\sigma_{\lambda, \omega} \gets \lambda\sigma_\omega$
    \EndWhile
    
    \While{$\lambda < \pi$}
        \State $\Lambda^+ \gets \Lambda^+ \cup \{\lambda\}$
        \State $\lambda \gets \lambda 2^{\frac{1}{Q}}$
        \State $\sigma_{\lambda, \omega} \gets \lambda\sigma_\omega$
    \EndWhile
\end{algorithmic}
\end{algorithm}

Only positive $\lambda$'s have been defined thus far, which provides inadequate coverage of the frequency domain in multiple dimensions. For real input signals, it is only necessary to cover half of one of the dimensions (only positive $\lambda$'s), whereas full coverage (both negative and positive $\lambda$'s) is required for additional dimensions. On-axis coverage is also required, in which each $\uppsi_\lambda$ must be multiplied with a Gaussian (zero-frequency wavelet), which the dilated wavelet definition in equation (\ref{eqn:dilwav}) defines as $\lambda=0$. A similar construction procedure is followed in \citep{jointtfscattering2}.

Given $m$ 1D filterbanks, with $m \ge 2$, each having Morlet filters with a positive set of lambdas $\Lambda_i^+$ and invariance scales $\vect{d} \in \mathbb{N}^m$, with $i = 1, ..., m$ indexing the dimension, we construct the $m$-dimensional filterbank with 
\begin{equation}
    \mathbb{F} = \left\{ \uppsi_{\vect{\lambda} }(\mathbf{u}) \ \middle| \ \vect{\lambda} \in \mathbb{L} \setminus \{\vect{0}\} \right\},
\end{equation}
where $\vect{u}$ is the $m$-dimensional spatial and/or time variable in which the each 1D filterbank is defined.  By definition, 
\begin{gather}
    \mathbb{L} = (\Lambda_1^+ \!\cup\! \{0\})\!\times\! (\Lambda_2^+ \!\cup\! \Lambda_2^- \!\cup \!\{0\}) \!\times\! ... \!\times\! (\Lambda_m^+ \!\cup\! \Lambda_m^- \!\cup\! \{0\}); \\
    \Lambda_i^- = \left\{-\lambda \ \middle| \ \lambda \in \Lambda_i^+\right\},
\end{gather}
where $\cup$ indicates the set union operator and $\times$ the Cartesian product. For $m=1$, the provided definitions result in a conventional \ac{1d} scattering transform \citep{1dscattering1}.

The $m$-dimensional \ac{lpf} is defined as 
\begin{equation}
    \phi(\vect{u}) = \uppsi_{\vect{0}}(\vect{u}).
\end{equation}

% The dilated wavelet definition provided by (\ref{eqn:dilwav}) therefore provides a concise set construction, with which to construct the \ac{lpf} and on-axis wavelets simultaneously.

% All on-axis filters will be highly correlated with their negative counterpart due to the symmetry of the multidimensional Fourier transform of real signals. For example, the 2D filter specified by $\lambda = [0, x]^T$ will be highly correlated with the filter at $[0, -x]^T$. As the invariance scale $d$ increases, the on-axis filters will become more narrow in the zero coordinate dimension, thereby becoming increasingly correlated.


% \subsection{Discretisation}

% Morlets do not have compact support, as such, they must be truncated for computation. The invariance scale $d$ specifies the maximum allowable time support of all filters. Since input signals must be at least of length $d$, filters can be precomputed using the padded signal length, while also guaranteeing a suitable level of truncation. The padding required is on the order of $d$ samples.

% Note that, due to their separability, all $\uppsi_{\vect{\lambda}} \in \mathbb{F}$ can be stored as $d$ one-dimensional filters, and not as $N_1 \times ... \times N_d$ tensors, thereby saving on the storage required for pre-computed filters. The savings on storage space allow implementations to compute the filters at multiple input sample frequencies, allowing for optimal downsampling strategies without incurring a significant storage cost.

\section{Separable Scattering Transform} \label{sec:wst}
% In this section, we describe the efficient computation of $m$-dimensional separable scattering transforms. The described downsampling strategy is utilised to a limited extent in Kymatio scattering implementations \citep{kymatio} and the original proposed fast algorithms \citep{2dscattering}, which is generalised further in this section.

\subsection{Transform}

The scattering transform requires 2 steps to provide scattering coefficients. The scalogram operator $\mathcal{U}_j$ iteratively filters a discrete signal $x[\vect{n}]$ for a given set of filters $\mathbb{F}$, which is then averaged by the \ac{lpf}. $\vect{n}$ represents a multidimensional index variable.
\begin{gather}
    \mathcal{U}_j x [\vect{n}, \vect{\lambda}_1, ..., \vect{\lambda}_j] = \left|\left(\mathcal{U}_{j-1} \ ...\  \mathcal{U}_1 x\right) * \uppsi_{\vect{\lambda}_j} \right|, \ \forall \ \uppsi_{\vect{\lambda}_j} \in \mathbb{F}; \\
     \mathcal{U}_1 x [\vect{n}, \vect{\lambda}_1] =  \left| x * \uppsi_{\vect{\lambda}_1} \right| , \ \uppsi_{\vect{\lambda}_1} \in \mathbb{F}.
\end{gather}

The scattering operator $\mathcal{S}_j$ provides the output coefficients at the $j$'th order of the scattering transform:
\begin{equation}
    \mathcal{S}_j x[\vect{n}, \vect{\lambda}_1, ..., \vect{\lambda}_j] = \mathcal{U}_j x * \phi.
\end{equation}
The \ac{lpf} $\phi$ remains constant throughout the transform. Note that the modulus/magnitude operator $|\cdot|$ demodulates the output of the filters, effectively extracting the Hilbert envelope from a band-limited signal \citep{waveletsandsubbandcoding}. 

The $j$'th scattering order adds an additional axis of paths indexed by $\vect{\lambda}_j$. However, not all paths need to be evaluated, since some paths have smaller bandwidths, thereby requiring fewer filters to extract the information lost by averaging. In particular, we only evaluate paths in which the centre frequencies of all elements of the vector $\vect{\lambda}_{j}$ are smaller than their corresponding bandwidth of the previous path's filter $\uppsi_{\vect{\lambda}_{j-1}}$. Path pruning is therefore dependent on $\alpha$ and $\beta$.

\subsection{Downsampling Strategy}
Since each filter specified by $\vect{\lambda}$ has its own bandwidth, we can employ downsampling across all paths non-uniformly. However, some care is required to ensure the compounded downsampling steps across all paths result in a uniform sampling frequency of the output scattering coefficients.

% The invariance scale $d$ is equivalent to the total downsampling required to compute scattering coefficients. $d$ is typically chosen as a power of $2$, which allows for multiple decimation stages to also subsample by powers of $2$. This decimation scheme is illustrated particularly well by dyadic wavelet transforms.

% However, it is not required that $d$ is restricted to a power of $2$, but rather that compounded decimation steps result in an effective decimation of $d$. As such, $d$ is most effective when it has as many factors as possible. In this study, we will require that $d$ is an even number.

Many applications are insensitive to small changes in $d$. As such, we propose a strategy to find an optimal $d$ given a target and tolerance value. For some applications, choosing $d$ such that the downsampling factor is a power of 2 is the simplest solution to achieve optimality. 

Without prior knowledge of the filterbank configurations, given a target invariance scale of $\bar{d}$ samples and a tolerance $\epsilon$, we can optimise $d \in \{\floor{(1 - \epsilon)\bar{d}}, \ceil{(1 + \epsilon)\bar{d}}\}$ such that it results in largest number of supported downsampling configurations.

A downsample factor $d$ which decomposes into a set of $n$ prime factors $\{p_1, ..., p_n\}$ with a corresponding multiset $\mathbb{M} = \left\{ m_1, ..., m_n \right\}$, where $m_i$ is the multiplicity of the prime $p_i$. We can find an optimal $d$ by maximising the sum $\sum\limits_{m \in M} m$.

% However, the proposed scheme for finding an optimal $d$ allows for more choices, thereby no restricting the choice of $d$ much, while also allowing for efficient dowmsampling.

Morlet filters in a 1D filterbank may be downsampled by a factor $d_{\psi_1}$ prior to low-pass filtering, and then downsampling again by a factor $d_{\phi_1}$ after low-pass filtering. As such, the compounded effect of downsampling restricts $d = d_{\psi_1} \cdot d_{\phi_1}$.

In the second order of scattering, the process is repeated with an additional pre-low-pass downsampling factor of $d_{\psi_2}$. The second level application of $\phi$ then downsamples by a factor $d_{\phi_2}$. To maintain a consistent output sampling frequency, it restricts $d = d_{\psi_2} \cdot d_{\psi_1} \cdot d_{\phi_2}$. 

Continuing the downsampling chain, the $i$'th level of downsampling requires $d = d_{\psi_i} \cdot ... \cdot d_{\psi_1} \cdot d_{\phi_i}$. The output of each operation of $\mathcal{U}_i$ and $\mathcal{S}_i$ must be downsampled as much as possible in order to make subsequent operations faster. To ensure that the application of all the downsampling steps are efficient, we require $d$ to have as many prime factors as possible, including factor multiplicity, so that a larger variety of downsampling combinations may be supported.



% \begin{algorithm}    
% \caption{Finding an optimal $d$.}\label{alg:optt}
% \begin{algorithmic}
%     \State $d_1 \gets \floor*{\frac{f_s \bar{T}(1 - \epsilon)}{2 \beta}}$\\
%     \State $d_2 \gets \ceil*{\frac{f_s \bar{T}(1 + \epsilon)}{2 \beta}}$\\
%     \State $k_\text{max} \gets 1$\\
%     \For{$d_j \in \{d_1, d_1 + 1, \ ... \ , d_2 - 1, d_2\}$}
%         \State $M \gets \text{PrimeMultiplicity}(d_j)$
%         \State  $k \gets \sum\limits_{m \in M} m$\\
%         \If{$k > k_\text{max}$}
%             \State $k_\text{max} \gets k$
%             \State $d \gets d_j$
%         \EndIf
%     \EndFor
% \end{algorithmic}
% \end{algorithm}


Consider a 1D wavelet filterbank and a single wavelet filter $\uppsi_1[n]$, applied to a discrete-time signal $x$. The operations required to compute the scattering coefficients is then notated for simplicity as
\begin{gather}
    u_1[n] = \left|  x * \uppsi_1   \right|; \\
    s_1[n] = y_1 * \phi.
\end{gather}

The bandwidth of $u_1$ is the bandwidth of an arbitrary first filter $\uppsi_1$. This follows from the Hilbert envelope computed by the analytic wavelet filter $\uppsi_\lambda$ and the modulus $|\cdot |$. The bandwidth of $s_1$ is the bandwidth of $\phi$.

Critical downsampling of a wavelet $\uppsi$ with a bandwidth of $\sigma_\omega$ is achieved by a factor of $d_{\uppsi} = \floor*{\frac{ \pi }{\beta\sigma_\omega}}$. Critical downsampling of $s_1$, is achieved via a factor of $d$, by definition.

We can efficiently compute $s_1$ using compounded downsampling steps:
\begin{equation}
    \left(s_1\right)_{\downarrow d} = \left( \left(u_1\right)_{\downarrow d_1} * (\phi)_{\downarrow d_1}\right)_{\downarrow d_2},
\end{equation}
such that $d = d_1 \cdot d_2, \ d_1, d_2 \in \mathbb{N}^+$, with $d_1 | d$ and $d_1 \le  d_{\uppsi_1}$. In order to find $d_1$, we decrement $d_{\uppsi_1}$ until it divides $d$ evenly. Each scalogram $u_j[n]$ is not necessarily downsampled optimally, but has a downsampling factor which guarentees a consistent scattering coefficient output sample frequency. 

% This strategy may counter intuitively result in faster computations for filterbanks with more filters, since a higher $Q$ implies a smaller bandwidth of each filter. 

A second order of scattering with a filter path of $(\uppsi_1, \uppsi_2)$ is performed on the downsampled $u_1$:
\begin{gather}
    u_2 = \left|  \left(u_1\right)_{\downarrow d_1} * (\uppsi_2)_{\downarrow d_1}  \right|; \\
    \left(s_2\right)_{\downarrow d} = \left( \left(u_2\right)_{\downarrow d_3} * (\phi)_{\downarrow d_3}\right)_{\downarrow d_4},
\end{gather}
such that $d = d_1 \cdot d_3 \cdot d_4, \ d_1, d_3, d_4 \in \mathbb{N}^+, d_3 \le d_{\uppsi_2}, d_3 | \frac{d}{d_1}$. The proposed downsampling scheme can be extended to an arbitrary number of scattering orders.

% Since the filterbank configuration is known, all required combinations of downsampled filters and their downsampling factors can be precomputed.




\subsection{Convolutions}
Optimal \ac{fft} convolutions can be achieved by performing downsampling in the frequency domain instead of the time domain. It is straightforward to verify that $|x * \uppsi|_{\downarrow r} = \left|(x * \uppsi)_{\downarrow r}\right|$, since the modulus is an element-wise operation, for some downsampling factor $r$. Given the signal and filter Fourier transforms $\hat{x}$ and $\hat{\uppsi}$, we then have
\begin{equation}
\label{eqn:circconv}
    x \otimes \uppsi [n] \xleftrightarrow{\mathcal{FFT}}  \hat{x} \cdot \hat{\uppsi} [k],
\end{equation}
where the $\otimes$ operator represents a circular convolution and $k$ is the frequency index.

Given that $r|N$, we can express (\ref{eqn:circconv}) when downsampled as a periodised summation \citep{waveletsandsubbandcoding} in the frequency domain
\begin{equation}
    (x \otimes \uppsi [n])_{\downarrow r} \xleftrightarrow{\mathcal{FFT}}  \frac{1}{r} \sum_{i=0}^{r-1} \hat{x} \cdot \hat{\uppsi} [k + iN/r], %https://citeseerx.ist.psu.edu/document?repid=rep1&type=pdf&doi=5216ea733e562541b33a7f97dab0de072b2e8827
\end{equation}
which can be efficiently implemented via shape manipulation of tensors in computational packages like MATLAB or PyTorch \citep{pytorch}. To compute valid convolutions, we must pad $x$ and $\uppsi$ to have a total length of $N = N_x + d + c$, where $c \in \mathbb{N}^+$ is a constant that ensures that $d | N$. 

% Note that only $d$ additional samples are required instead of $2d$, since $d$ is, by definition, the time support of $\phi$ in number of samples. Padding is applied equally to the left and right of $x$. For further reduction of boundary effects, reflection padding is utilised. Kymatio's 2D scattering implementation utilises a similar padding scheme. 

% Thus far, the techniques discussed may be equivalently implemented on many scattering transforms. Separable scattering benefits from reduced computations resulting from filter separability. In particular, given a $m$-dimensional signal $x[\vect{n}]$ indexed by $\vect{n} = (n_1, ..., n_m)^T$, with each dimension having downsample factors expressed as a vector $\vect{r} = (r_1, ..., r_m)^T$, we can chain downsampling for each dimension:
% \begin{equation}
%     \left(x * \uppsi_{\vect{\lambda}}[\vect{n}]\right )_{\downarrow \vect{r}} = \left( \left(x[\vect{\lambda}] * \uppsi_{\lambda_i}[n_i]\right)_{\downarrow r_1} ... * \uppsi_{\lambda_m}[n_m]\right)_{\downarrow r_m}.
% \end{equation}.

% We abuse notation to indicate that $\downarrow r_i$ is an operator that downsamples the $i$'th dimension, and $\downarrow \vect{r}$ downsamples all dimensions each with their correponding downsample factor in $\vect{r} \in \mathbb{N}^m$.

% For $m$-dimensional sequences of dimensional lengths $N_1, ..., N_m$, the forward and inverse \acp{fft} computation time is reduced from $2\sum_{i} T \log N_i$ (no frequency periodisation), with $T = \prod_i N_i$ to $\sum_{i} \frac{T}{K_i} \log \frac{N_i}{K_i} + P \log P$, where $K_1 = 1$, $K_i = \prod_{j < i} r_j$ and $P = \prod_i \frac{N_i}{r_i}$. However, the number of convolution multiplications in the frequency domain increases from $\prod_i N_i$ to $\sum_i \prod_j \frac{N_j}{K_i} \leq m \prod_i N_i$.

% In contrast, optimal non-separable scattering transforms can achieve a computational cost of $\sum_{i} N_i \log N_i + P \log P$, since a full forward \ac{fft} is required to compute a convolution with a non-separable $m$-dimensional filter.

% The greatest computational advantage is during direct convolution computations, as is present in \acp{cnn}. For an $m$-dimensional filter impulse response with filter lengths $L_1, ..., L_m$, the computational cost reduces from $P\prod_i L_i$ to $P \sum_i L_i$.


\section{Results}



\subsection{MNIST}

% We benchmark separable \ac{ws} against 2D scattering by repeating the MNIST hand-written digit dataset experiment in \citep{2dscattering}. 

The MNIST dataset \citep{mnist} contains 60000 training and 10000 test samples. Unlike in \citep{2dscattering}, which decorrelates scattering coefficients with a discrete cosine transform prior to classification, we perform classification on the scattering coefficients directly. Kymatio \citep{kymatio} is used to produce the 2D scattering coefficients. Our implementation of separable scattering is implemented similarly to Kymatio, with PyTorch \citep{pytorch} as a backend for FFT convolutions and \ac{nn} models.

Unless specified otherwise, all experiments have $\beta = \alpha = 2.5$. Scattering features are normalised prior to classification, according to the mean and \ac{std} calculated on the training set. No data augmentation is performed.

\begin{table}
\centering
\caption{MNIST classification error rate (\%) of separable and \ac{2d} scattering coefficients using a \ac{nn} classifier} \label{tab:mnistnn}
\begin{tabular}{|lr|l|} \hline
\multirow{2}{*}{2D WS + NN}        & $l=1$, $J=2$  & $0.64 \pm 0.05$    \\
          & $l=2$, $J=3$ & $0.50 \pm 0.03$     \\ \hline
\multirow{2}{*}{Separable WS + NN}  & $l=1$, $d=(4,4)$ & $0.63 \pm 0.05 $    \\
          & $l=2$, $d=(4,4)$  & $0.52 \pm 0.04$     \\ \hline
\end{tabular}
\end{table}


% Table \ref{tab:mnistlda} shows the results for MNIST using a \ac{lda} classifier from the Scikit-learn python package \citep{sklearn}, with a covariance shrinkage factor of $5\cdot 10^{-3}$. Scattering is performed for $l=1, 2$ levels.

% \begin{table}[!h] 
% \centering
% \caption{MNIST classification error rate (\%) of separable and \ac{2d} scattering coefficients  using an \ac{lda} classifier} \label{tab:mnistlda}
% \begin{tabular}{|r|rr|rr|} \hline
% \multirow{2}{*}{Training size}              & \multicolumn{2}{l}{2D}                               & \multicolumn{2}{l|}{Separable}                             \\
%  & \multicolumn{1}{l}{$l=1$} & \multicolumn{1}{l}{$l=2$} & \multicolumn{1}{l}{$l=1$} & \multicolumn{1}{l|}{$l=2$} \\ \hline
% 1000                              & 4.58                        & 2.69                        & 4.87                        & 4.58                        \\
% 2000                              & 3.89                        & 1.62                        & 4.31                        & 2.6                         \\
% 5000                              & 3.54                        & 1.12                        & 3.84                        & 1.7                         \\
% 10000                             & 3.35                        & 1.05                        & 3.54                        & 1.5                         \\
% 20000                             & 3.16                        & 0.87                        & 3.41                        & 1.41                        \\
% 40000                             & 2.92                        & 0.86                        & 3.34                        & 1.25                        \\
% 60000                             & 2.92                        & 0.81                        & 3.27                        & 1.34     \\ \hline                  
% \end{tabular}
% \end{table}

Due to its separability and non-angularly spaced filters, separable scattering does not perform as well compared to 2D scattering when using simple classifiers, such as \ac{lda} \citep{lda}. To illustrate that this performance discrepancy is not of significant consequence in a \ac{nn} setting, we test performance on the full dataset utilising a simple architecture. The neural network architecture used has an input layer with 256 neurons, followed by two hidden layers with 128 and 64 neurons respectively. The output layer has 10 neurons - one for each digit. Input and hidden layers are followed by a batch norm layer \citep{batchnorm} and ReLU activation function \citep{relu}. The output layer is followed by a softmax function. The Adam optimiser \citep{adam} is used with cross-entropy loss, a batch size of 256 and learning rate of $3 \cdot 10^{-5}$. 5000 of the 60000 training samples are reserved for validation and removed from the training set. Training is stopped when validation loss starts to increase. The \ac{nn} model is initialised with random weights, and the experiment is repeated 50 times. Different invariance scales were tested, and the best results are reported in table \ref{tab:mnistnn}. Tests are repeated for $l \in \{1, 2\}$ levels of scattering.





\subsection{MedMNIST3D}


The MedMNIST3D datasets are a subset of the MedMNIST dataset group \citep{medmnist}, where each 3D dataset contains $28\times 28 \times 28$ images with 2, 3 or 11 classes. Train, test and validation data partitions are provided by the authors. All datasets have on the order of 1000 training samples. We compare the baseline \ac{nn} results provided in \citep{medmnist} with separable scattering features combined with a simple \ac{nn} classifier.

We use \ac{nn} classifier with an input layer containing 1024 neurons, followed by two hidden layers with 512 and 256 neurons respectively. Input and hidden layers layers are each followed by a batch norm layer \citep{batchnorm} and a ReLU non-linearity \citep{relu}. For datasets with two classes, the output layer is a single neuron followed by a sigmoid activation function. For datasets with more than two classes, the output layer has a size equal to the number of classes, followed by a softmax activation. Binary cross-entropy loss are used for datasets with two classes, otherwise cross-entropy loss is used. The Adam optimiser \citep{adam} with a learning rate of $1 \cdot 10^{-5}$ is used. All other configuration parameters are identical to the model used for the MNIST dataset.

A single level of scattering coefficients are computed, with $Q = (2, 2, 2)$ and $d = (4, 4, 4)$. Many MedMNIST3D datasets tend to be unbalanced, implying that \ac{auc} is a more reliable metric to measure model performance. The results are shown in table \ref{tab:medmnist}, in which \ac{sota} \ac{auc} performance is achieved for the Organ, Adrenal and Vessel datasets. Table \ref{tab:medmnist} indicates the number of classes ($c$) for each of the datasets, with accuracy (ACC) also shown for reference. The performance of our method on non-\ac{sota} results are comparable to the other baseline \ac{nn} approaches presented in \citep{medmnist}. It is likely that better results can be achieved by the proposed method if the filters are made learnable and/or scattering parameters are uniquely optimised for each dataset.

\begin{table*}[t!]
    \centering
    \caption{MedMNIST classification results of compared to baseline \ac{nn} approaches (\citep{medmnist})} \label{tab:medmnist}
    \begin{adjustbox}{angle=90}
\begin{tabular}{|l|rr|rr|rr|rr|rr|rr|}
\hline
\multirow{2}{*}{Methods}                & \multicolumn{2}{c}{Organ ($c=11$)}                  & \multicolumn{2}{c}{Nodule ($c=2$)}                 & \multicolumn{2}{c}{Fracture ($c=3$)}               & \multicolumn{2}{c}{Adrenal ($c=2$)}                & \multicolumn{2}{c}{Vessel ($c=2$)}                 & \multicolumn{2}{c|}{Synapse ($c=2$)}                \\
                         & \multicolumn{1}{l}{AUC} & \multicolumn{1}{l}{ACC} & \multicolumn{1}{l}{AUC} & \multicolumn{1}{l}{ACC} & \multicolumn{1}{l}{AUC} & \multicolumn{1}{l}{ACC} & \multicolumn{1}{l}{AUC} & \multicolumn{1}{l}{ACC} & \multicolumn{1}{l}{AUC} & \multicolumn{1}{l}{ACC} & \multicolumn{1}{l}{AUC} & \multicolumn{1}{l|}{ACC} \\ \hline
ResNet-1810+2.5D        & 0.977                   & 0.788                   & 0.838                   & 0.835                   & 0.587                   & 0.451                   & 0.718                   & 0.772                   & 0.748                   & 0.846                   & 0.634                   & 0.696                   \\
ResNet-1810+3D          & 0.996                   & 0.907                   & 0.863                   & 0.844                   & 0.712                   & 0.508                   & 0.827                   & 0.721                   & 0.874                   & 0.877                   & 0.82                    & 0.745                   \\
ResNet-1810+ACS41       & 0.994                   & 0.900                     & 0.873                   & 0.847                   & 0.714                   & 0.497                   & 0.839                   & 0.754                   & 0.930                    & 0.928                   & 0.705                   & 0.722                   \\
ResNet-5010+2.5D        & 0.974                   & 0.769                   & 0.835                   & 0.848                   & 0.552                   & 0.397                   & 0.732                   & 0.763                   & 0.751                   & 0.877                   & 0.669                   & 0.735                   \\
ResNet-5010+3D          & 0.994                   & 0.883                   & 0.875                   & 0.847                   & 0.725                   & 0.494                   & 0.828                   & 0.745                   & 0.907                   & 0.918                   & \textbf{0.851}          & \textbf{0.795}          \\
ResNet-5010+ACS41       & 0.994                   & 0.889                   & 0.886                   & 0.841                   & \textbf{0.750}           & \textbf{0.517}          & 0.828                   & 0.758                   & 0.912                   & 0.858                   & 0.719                   & 0.709                   \\
auto-sklearn11          & 0.977                   & 0.814                   & \textbf{0.914}          & \textbf{0.874}          & 0.628                   & 0.453                   & 0.828                   & \textbf{0.802}          & 0.910                   & \textbf{0.915}          & 0.631                   & 0.730                    \\
AutoKeras12             & 0.979                   & 0.804                   & 0.844                   & 0.834                   & 0.642                   & 0.458                   & 0.804                   & 0.705                   & 0.773                   & 0.894                   & 0.538                   & 0.724                   \\ \hline
Separable WS + NN (Ours)  & \textbf{0.998}          & \textbf{0.941}          & 0.858                   & 0.797                   & 0.614                   & 0.458                   & \textbf{0.875}          & 0.792                   & \textbf{0.962}          & 0.895                   & 0.715                   & 0.525    \\ \hline              
\end{tabular}
\end{adjustbox}
\end{table*}



\chapter{Wavelet Scattering in Higher Dimensions}


\chapter{Separable Wavelet Scattering}
\label{chap:p3}




\articleinfo{M.W. Rademan, D.J.J Versfeld, J.A. Du Preez}{IEEE Signal Processing Letters}{15 May 2024}{doi}{Wavelet scattering is a widely used feature extraction method due to it efficacy in extracting invariant features, 
while retaining any lost high frequency information resulting from averaging to obtain the desired amount of invariance. 
By generalising a 1-dimensional scattering transform, we extend its definition to an arbitrary number of independent dimensions. 
We show that, in a modern deep learning setting, separable wavelet scattering performs similarly to its non-separable counterparts 
with the MNIST hand-written digit dataset. We additionally demonstrate state-of-the-art results for a subset of the MedMNIST3D datasets.}


\section{Introduction}
\Ac{ws} has seen widespread use in classification applications as a powerful feature extraction method. It is an especially effective method for small datasets, since the feature filters are reminiscent of a \ac{cnn} front-end, while requiring no learning. The resulting features are invariant across all averaged dimensions, and exhibit separable class subspaces, allowing linear classifiers to be used with high efficacy \citep{2dscattering}.

Various forms of \ac{ws} exist, with \ac{1d} scattering first proposed by Anden and Mallat \citep{1dscattering1, ws}, which was later extended to 2 \citep{2dscattering} and 3 dimensions \citep{3dscattering, harmonicscattering}. Additional and more specialised forms for \ac{ws} include rotation-invariant scattering \citep{groupinvariantscattering} for the classification of textures and joint-\ac{tf} scattering \citep{ws_joint_tf, jointtfscattering2}. Joint-\ac{tf} scattering utilises a separable \ac{2d} filter that operates on the first level scattering scalogram, denoted by the operator $\mathcal{U}_1$, which is the only apparent usage of separable filters in the current scattering literature. 

In a deep-learning setting, sensible initialisation of filters prior to learning can significantly improve performance and interpretability \citep{sincnet}. The greatest advantage of separable filters is their computation speed when computing convolutions directly, as is performed in \acp{cnn} \citep{separablecnn}. Additionally, separable filters reduce the number of parameters of a \ac{cnn} if the filters are configured to be learnable. Learnable filters can typically improve performance compared to their fixed-filters counterpart \citep{scattering_birdsong}. 

At first glance, a \ac{ws} decomposition seems to be prohibitively expensive, but fast algorithms are possible due to the limited bandwidth of filter output. Fast algorithms utilise downsampling to take advantage of the demodulation of band-limited signals that result from the convolution of analytic wavelets \citep{2dscattering, 3dscattering, 1dscattering1}. Path pruning is also used to ignore filter combinations which have negligible energy. These algorithms are the standard implementation on many platforms, such as the Kymatio python package \citep{kymatio} and MATLAB \citep{MATLAB}.

The Morlet wavelet is the most widely used wavelet filter for a scattering filter bank implementation. Since the Morlet does not have compact support, implementations vary when considering the bandwidth/time support of the Morlet. This affects how Morlet \ac{fir} filters are discretised and truncated, how scattering paths with negligible energy are pruned, and how filters overlap in the frequency domain. 

In this work, we generalise \ac{1d} scattering to an arbitrary number of dimensions, which we refer to as the separable scattering transform. We propose flexible definitions for the Morlet bandwidth and filter overlap, which allows for an exact definition of near-optimal convolution computations with downsampling. Our implementation of separable scattering is \ac{gpu} accelerated, similar to the Kymatio implementations \citep{kymatio}. 

We demonstrate using the MNIST handwritten digit dataset \citep{mnist} that separable wavelet filters still perform adequately in a \ac{nn} classification setting compared to the conventional 2D scattering filters. We demonstrate the arbitrarily scalable dimensionality of the separable scattering transform with three-dimensional medical datasets from MedMNIST3D \citep{medmnist}. Separable scattering coefficients show \ac{sota} results for some of the MedMNIST3D datasets when combined with a simple \ac{nn}, while performing comparably to \ac{cnn} benchmarks on datasets which do not indicate \ac{sota} results.

\section{Separable Morlet Filterbank}

% In this section, we consider the continuous Morlet and provide general definitions of bandwidth and time-support for multiple dimensions. We utilise the Gaussian window to construct $n$-dimensional Morlets of varying characteristics.

\subsection{Morlet}

We define a \ac{1d} zero-mean Gaussian as
\begin{equation}
    \theta_{\sigma_t}(t) = \frac{1}{\sqrt{2\pi\sigma_t^2}}e^{-\frac{1}{2}\left(\frac{t}{\sigma_t}\right)^2},
\end{equation}
which has a Fourier transform transform $\theta_{\sigma_t}(t) \xleftrightarrow{\mathcal{F}} \hat{\theta}_{\sigma_\omega}(\omega)$, where $\sigma_\omega = \frac{1}{\sigma_t}$.

For an arbitrary bandwidth definition, we define the bandwidth-to-$\sigma$ ratio $\beta \in \mathbb{R}^+$, such that the one-sided bandwidth of a zero-mean Gaussian is $\beta \sigma_\omega$. For reference, Kymatio's implementation of 2D scattering indirectly defines $\beta \approx 2.5$ \citep{kymatio}.

% The $-3$ dB point of a Gaussian \ac{lpf} occurs at $\beta \approx 0.833$, but since the Gaussian \ac{lpf} does not have a very steep cutoff, values of $\beta \in [1, 3]$ may be appropriate to combat aliasing. 

A \ac{1d} Morlet $\uppsi$ has energy concentrated around 1 rad/s, with most of its energy contained in the interval $\omega \in [1 - \frac{1}{Q}, 1 + \frac{1}{Q}]$. $Q \in \mathbb{R}^+$ is defined as the number of wavelets per octave. The mother wavelet is given as
\begin{equation}
    \uppsi(t) = \theta_{\sigma_t}(t)\left(e^{jt} - \theta_0 \right),
\end{equation}
where $\theta_0 = \frac{\theta_{\sigma_t}(-1)}{\theta_{\sigma_t}(0)}$ to ensure zero mean: $\hat{\uppsi}(0) = 0$.

For reasons that will become apparent in section \ref{sec:filterbank}, we defined a 1D-wavelet dilated by a factor $\lambda$ as 
\begin{equation}
\label{eqn:dilwav}
    \uppsi_\lambda(t) = \begin{cases}
        \uppsi(\lambda t), \ \lambda \neq 0 \\
        \phi_j(t), \ \lambda = 0
    \end{cases},
\end{equation}
where $\phi$ represents the 1D \ac{lpf} utilised in scattering computations for the current dimension ($j$).

A $n$-dimensional separable wavelet may be constructed similarly, with
\begin{equation}
    \uppsi_{\vect{\lambda} }(\mathbf{u}) = \prod_{i=0}^{n-1} \uppsi_{\lambda_i}(u_i),
\end{equation}
where $\mathbf{u} = \left(u_1, ..., u_n\right)^T$ is the vector containing the dimensions of interest.

A filterbank is constructed by dilating the mother wavelet with a set of dilation factors $\vect{\lambda} =(\lambda_1, ..., \lambda_n )^T \in \Lambda_1 \times ... \times \Lambda_n$, with $\Lambda_i$ the set of dilation factors for the variable $u_i$. The dilation factor $\lambda_i$ is also the centre frequency of the dilated wavelet in rad/s. 

% Each dimension indexed by $i$ has its own defined \ac{lpf}.

% with with a dilated one-sided bandwidth of $\lambda_i \beta \sigma_{\omega_i}$. 

\subsection{1D Filterbank Construction}
\label{sec:filterbank}
 Suppose a 1D Morlet has a centre frequency $\lambda_0$. The following filter is placed at $\lambda_0 2^{\frac{1}{Q}}$. To define the amount of overlap between filters, it is useful express the corresponding filter's frequency \ac{std} $\sigma_\omega$ in terms of the distance between filters in the frequency domain. If the wavelet at $\lambda_0$ requires a decay equivalent to $\alpha \sigma_\omega$ \ac{std}s at the next wavelet at $\lambda_0 2^{\frac{1}{Q}}$. It follows that $\sigma_\omega = \frac{1}{\sigma_t} = \frac{1}{\alpha}\left( 2^\frac{1}{Q} - 1 \right)$.

We refer to $\alpha$ as the clearance factor. More overlap will result in more redundancy in the frequency representation, but also shorter filter impulse responses. 

% It is possible to set the clearance factor equal to the defined Gaussian bandwidth: $\alpha = \beta$. However, different values of $\alpha$ and $\beta$ allow for the separate definition of filter clearance and bandwidth, allowing the application to tune downsampling and filter time support separately. 

The provided definitions of filter clearance ($\alpha$) and bandwidth ($\beta$) may also be extended to non-separable wavelets, such as the rotational construction employed in \citep{2dscattering}. However, a conversion is required to find the bandwidth in each axis for a specific wavelet for rotationally constructed filterbanks, in order to employ the downsampling strategies proposed in this paper.

% A \ac{1d} wavelet filterbank is constructed with a set of filters $\mathbb{F} := \left\{\uppsi_\lambda \middle| \lambda \in \Lambda\right\}$ and a \ac{lpf} $\phi$ with the following specified parameters:
% \begin{enumerate}
%     \item $Q$ - wavelets per octave.
%     \item  $\omega_0$ - starting frequency in rad/s (optional).
%     \item  $d$ - invariance scale in samples, equivalent to the decimation factor applied to scattering coefficients.
% \end{enumerate}

All filterbank descriptions in this paper utilise normalised sampling frequency, i.e., the sample frequency is $f_s = 1$.

For some audio applications, for example speech recognition, low-frequency content is negligible, motivating filterbank construction starting at a specified frequency $\omega_0$. Otherwise, the entire frequency domain may be covered by setting $\omega_0$ to start at a position specified by $\phi$ and $\alpha$.

The \ac{lpf} $\phi$ is constant across all layers in a \ac{1d} scattering transform, and is chosen as $\phi(t) = \theta_{\sigma_{\phi, t}}(t)$, where $\sigma_{\phi, t} = \dfrac{d}{ \pi \beta}$.

The scattering transform requires that the time support of all filters do not exceed the time support of $\phi$, ensuring no filters contain time information exceeding the invariance scale $d$. $d$ is equivalently defined as the total downsample factor of the scattering tranform, and is unique for each dimension. Equivalently, the frequency \ac{std} of the filters may not exceed the \ac{std} of $\phi(\omega)$. For brevity, we denote the frequency \ac{std}s as $\sigma_{\lambda, \omega}$ and $\sigma_{\phi, \omega}$ for the Morlets and the \ac{lpf} respectively. 

 Since the maximum time support (minimum bandwidth) is a function of $d$, all dilated wavelets have their Gaussian envelopes restricted to a maximum time support of $\sigma_{\phi, t}$. This requires linearly spaced filters until the dilated bandwidth is larger than the \ac{lpf} bandwidth. To construct a set of positive dilation factors $\lambda \in \Lambda^+$, refer to algorithm \ref{alg:fb}.

\begin{algorithm}    
\caption{$\Lambda^+$ construction.}\label{alg:fb}
\begin{algorithmic}
    \State $\sigma_{\phi, \omega} \gets \dfrac{\pi\beta}{d}$
    \If{frequency limited}
        \Ensure{$\omega_0 \ge \dfrac{\pi\alpha}{d}$}
        \State $\lambda \gets \omega_0$
    \Else{}
        \State $\lambda \gets \dfrac{\pi\alpha}{d}$
    \EndIf
    \State $\Lambda^+ \gets \varnothing$
    \State $\sigma_\omega \gets \dfrac{1}{\alpha}\left( 2^\frac{1}{Q} - 1 \right)$
    \State $\sigma_{\lambda, \omega} \gets \lambda\sigma_\omega$
    
    \While{$\sigma_{\phi, \omega} > \sigma_{\lambda, \omega}$ and $\lambda < \pi$} 
        \State $\Lambda^+ \gets \Lambda^+ \cup \{\lambda\}$
        \State $\lambda \gets \lambda + \alpha \sigma_{\phi, \omega}$ 
        \State $\sigma_{\lambda, \omega} \gets \lambda\sigma_\omega$
    \EndWhile
    
    \While{$\lambda < \pi$}
        \State $\Lambda^+ \gets \Lambda^+ \cup \{\lambda\}$
        \State $\lambda \gets \lambda 2^{\frac{1}{Q}}$
        \State $\sigma_{\lambda, \omega} \gets \lambda\sigma_\omega$
    \EndWhile
\end{algorithmic}
\end{algorithm}

Only positive $\lambda$'s have been defined thus far, which provides inadequate coverage of the frequency domain in multiple dimensions. For real input signals, it is only necessary to cover half of one of the dimensions (only positive $\lambda$'s), whereas full coverage (both negative and positive $\lambda$'s) is required for additional dimensions. On-axis coverage is also required, in which each $\uppsi_\lambda$ must be multiplied with a Gaussian (zero-frequency wavelet), which the dilated wavelet definition in equation (\ref{eqn:dilwav}) defines as $\lambda=0$. A similar construction procedure is followed in \citep{jointtfscattering2}.

Given $m$ 1D filterbanks, with $m \ge 2$, each having Morlet filters with a positive set of lambdas $\Lambda_i^+$ and invariance scales $\vect{d} \in \mathbb{N}^m$, with $i = 1, ..., m$ indexing the dimension, we construct the $m$-dimensional filterbank with 
\begin{equation}
    \mathbb{F} = \left\{ \uppsi_{\vect{\lambda} }(\mathbf{u}) \ \middle| \ \vect{\lambda} \in \mathbb{L} \setminus \{\vect{0}\} \right\},
\end{equation}
where $\vect{u}$ is the $m$-dimensional spatial and/or time variable in which the each 1D filterbank is defined.  By definition, 
\begin{gather}
    \mathbb{L} = (\Lambda_1^+ \!\cup\! \{0\})\!\times\! (\Lambda_2^+ \!\cup\! \Lambda_2^- \!\cup \!\{0\}) \!\times\! ... \!\times\! (\Lambda_m^+ \!\cup\! \Lambda_m^- \!\cup\! \{0\}); \\
    \Lambda_i^- = \left\{-\lambda \ \middle| \ \lambda \in \Lambda_i^+\right\},
\end{gather}
where $\cup$ indicates the set union operator and $\times$ the Cartesian product. For $m=1$, the provided definitions result in a conventional \ac{1d} scattering transform \citep{1dscattering1}.

The $m$-dimensional \ac{lpf} is defined as 
\begin{equation}
    \phi(\vect{u}) = \uppsi_{\vect{0}}(\vect{u}).
\end{equation}

% The dilated wavelet definition provided by (\ref{eqn:dilwav}) therefore provides a concise set construction, with which to construct the \ac{lpf} and on-axis wavelets simultaneously.

% All on-axis filters will be highly correlated with their negative counterpart due to the symmetry of the multidimensional Fourier transform of real signals. For example, the 2D filter specified by $\lambda = [0, x]^T$ will be highly correlated with the filter at $[0, -x]^T$. As the invariance scale $d$ increases, the on-axis filters will become more narrow in the zero coordinate dimension, thereby becoming increasingly correlated.


% \subsection{Discretisation}

% Morlets do not have compact support, as such, they must be truncated for computation. The invariance scale $d$ specifies the maximum allowable time support of all filters. Since input signals must be at least of length $d$, filters can be precomputed using the padded signal length, while also guaranteeing a suitable level of truncation. The padding required is on the order of $d$ samples.

% Note that, due to their separability, all $\uppsi_{\vect{\lambda}} \in \mathbb{F}$ can be stored as $d$ one-dimensional filters, and not as $N_1 \times ... \times N_d$ tensors, thereby saving on the storage required for pre-computed filters. The savings on storage space allow implementations to compute the filters at multiple input sample frequencies, allowing for optimal downsampling strategies without incurring a significant storage cost.

\section{Separable Scattering Transform} \label{sec:wst}
% In this section, we describe the efficient computation of $m$-dimensional separable scattering transforms. The described downsampling strategy is utilised to a limited extent in Kymatio scattering implementations \citep{kymatio} and the original proposed fast algorithms \citep{2dscattering}, which is generalised further in this section.

\subsection{Transform}

The scattering transform requires 2 steps to provide scattering coefficients. The scalogram operator $\mathcal{U}_j$ iteratively filters a discrete signal $x[\vect{n}]$ for a given set of filters $\mathbb{F}$, which is then averaged by the \ac{lpf}. $\vect{n}$ represents a multidimensional index variable.
\begin{gather}
    \mathcal{U}_j x [\vect{n}, \vect{\lambda}_1, ..., \vect{\lambda}_j] = \left|\left(\mathcal{U}_{j-1} \ ...\  \mathcal{U}_1 x\right) * \uppsi_{\vect{\lambda}_j} \right|, \ \forall \ \uppsi_{\vect{\lambda}_j} \in \mathbb{F}; \\
     \mathcal{U}_1 x [\vect{n}, \vect{\lambda}_1] =  \left| x * \uppsi_{\vect{\lambda}_1} \right| , \ \uppsi_{\vect{\lambda}_1} \in \mathbb{F}.
\end{gather}

The scattering operator $\mathcal{S}_j$ provides the output coefficients at the $j$'th order of the scattering transform:
\begin{equation}
    \mathcal{S}_j x[\vect{n}, \vect{\lambda}_1, ..., \vect{\lambda}_j] = \mathcal{U}_j x * \phi.
\end{equation}
The \ac{lpf} $\phi$ remains constant throughout the transform. Note that the modulus/magnitude operator $|\cdot|$ demodulates the output of the filters, effectively extracting the Hilbert envelope from a band-limited signal \citep{waveletsandsubbandcoding}. 

The $j$'th scattering order adds an additional axis of paths indexed by $\vect{\lambda}_j$. However, not all paths need to be evaluated, since some paths have smaller bandwidths, thereby requiring fewer filters to extract the information lost by averaging. In particular, we only evaluate paths in which the centre frequencies of all elements of the vector $\vect{\lambda}_{j}$ are smaller than their corresponding bandwidth of the previous path's filter $\uppsi_{\vect{\lambda}_{j-1}}$. Path pruning is therefore dependent on $\alpha$ and $\beta$.

\subsection{Downsampling Strategy}
Since each filter specified by $\vect{\lambda}$ has its own bandwidth, we can employ downsampling across all paths non-uniformly. However, some care is required to ensure the compounded downsampling steps across all paths result in a uniform sampling frequency of the output scattering coefficients.

% The invariance scale $d$ is equivalent to the total downsampling required to compute scattering coefficients. $d$ is typically chosen as a power of $2$, which allows for multiple decimation stages to also subsample by powers of $2$. This decimation scheme is illustrated particularly well by dyadic wavelet transforms.

% However, it is not required that $d$ is restricted to a power of $2$, but rather that compounded decimation steps result in an effective decimation of $d$. As such, $d$ is most effective when it has as many factors as possible. In this study, we will require that $d$ is an even number.

Many applications are insensitive to small changes in $d$. As such, we propose a strategy to find an optimal $d$ given a target and tolerance value. For some applications, choosing $d$ such that the downsampling factor is a power of 2 is the simplest solution to achieve optimality. 

Without prior knowledge of the filterbank configurations, given a target invariance scale of $\bar{d}$ samples and a tolerance $\epsilon$, we can optimise $d \in \{\floor{(1 - \epsilon)\bar{d}}, \ceil{(1 + \epsilon)\bar{d}}\}$ such that it results in largest number of supported downsampling configurations.

A downsample factor $d$ which decomposes into a set of $n$ prime factors $\{p_1, ..., p_n\}$ with a corresponding multiset $\mathbb{M} = \left\{ m_1, ..., m_n \right\}$, where $m_i$ is the multiplicity of the prime $p_i$. We can find an optimal $d$ by maximising the sum $\sum\limits_{m \in M} m$.

% However, the proposed scheme for finding an optimal $d$ allows for more choices, thereby no restricting the choice of $d$ much, while also allowing for efficient dowmsampling.

Morlet filters in a 1D filterbank may be downsampled by a factor $d_{\psi_1}$ prior to low-pass filtering, and then downsampling again by a factor $d_{\phi_1}$ after low-pass filtering. As such, the compounded effect of downsampling restricts $d = d_{\psi_1} \cdot d_{\phi_1}$.

In the second order of scattering, the process is repeated with an additional pre-low-pass downsampling factor of $d_{\psi_2}$. The second level application of $\phi$ then downsamples by a factor $d_{\phi_2}$. To maintain a consistent output sampling frequency, it restricts $d = d_{\psi_2} \cdot d_{\psi_1} \cdot d_{\phi_2}$. 

Continuing the downsampling chain, the $i$'th level of downsampling requires $d = d_{\psi_i} \cdot ... \cdot d_{\psi_1} \cdot d_{\phi_i}$. The output of each operation of $\mathcal{U}_i$ and $\mathcal{S}_i$ must be downsampled as much as possible in order to make subsequent operations faster. To ensure that the application of all the downsampling steps are efficient, we require $d$ to have as many prime factors as possible, including factor multiplicity, so that a larger variety of downsampling combinations may be supported.



% \begin{algorithm}    
% \caption{Finding an optimal $d$.}\label{alg:optt}
% \begin{algorithmic}
%     \State $d_1 \gets \floor*{\frac{f_s \bar{T}(1 - \epsilon)}{2 \beta}}$\\
%     \State $d_2 \gets \ceil*{\frac{f_s \bar{T}(1 + \epsilon)}{2 \beta}}$\\
%     \State $k_\text{max} \gets 1$\\
%     \For{$d_j \in \{d_1, d_1 + 1, \ ... \ , d_2 - 1, d_2\}$}
%         \State $M \gets \text{PrimeMultiplicity}(d_j)$
%         \State  $k \gets \sum\limits_{m \in M} m$\\
%         \If{$k > k_\text{max}$}
%             \State $k_\text{max} \gets k$
%             \State $d \gets d_j$
%         \EndIf
%     \EndFor
% \end{algorithmic}
% \end{algorithm}


Consider a 1D wavelet filterbank and a single wavelet filter $\uppsi_1[n]$, applied to a discrete-time signal $x$. The operations required to compute the scattering coefficients is then notated for simplicity as
\begin{gather}
    u_1[n] = \left|  x * \uppsi_1   \right|; \\
    s_1[n] = y_1 * \phi.
\end{gather}

The bandwidth of $u_1$ is the bandwidth of an arbitrary first filter $\uppsi_1$. This follows from the Hilbert envelope computed by the analytic wavelet filter $\uppsi_\lambda$ and the modulus $|\cdot |$. The bandwidth of $s_1$ is the bandwidth of $\phi$.

Critical downsampling of a wavelet $\uppsi$ with a bandwidth of $\sigma_\omega$ is achieved by a factor of $d_{\uppsi} = \floor*{\frac{ \pi }{\beta\sigma_\omega}}$. Critical downsampling of $s_1$, is achieved via a factor of $d$, by definition.

We can efficiently compute $s_1$ using compounded downsampling steps:
\begin{equation}
    \left(s_1\right)_{\downarrow d} = \left( \left(u_1\right)_{\downarrow d_1} * (\phi)_{\downarrow d_1}\right)_{\downarrow d_2},
\end{equation}
such that $d = d_1 \cdot d_2, \ d_1, d_2 \in \mathbb{N}^+$, with $d_1 | d$ and $d_1 \le  d_{\uppsi_1}$. In order to find $d_1$, we decrement $d_{\uppsi_1}$ until it divides $d$ evenly. Each scalogram $u_j[n]$ is not necessarily downsampled optimally, but has a downsampling factor which guarentees a consistent scattering coefficient output sample frequency. 

% This strategy may counter intuitively result in faster computations for filterbanks with more filters, since a higher $Q$ implies a smaller bandwidth of each filter. 

A second order of scattering with a filter path of $(\uppsi_1, \uppsi_2)$ is performed on the downsampled $u_1$:
\begin{gather}
    u_2 = \left|  \left(u_1\right)_{\downarrow d_1} * (\uppsi_2)_{\downarrow d_1}  \right|; \\
    \left(s_2\right)_{\downarrow d} = \left( \left(u_2\right)_{\downarrow d_3} * (\phi)_{\downarrow d_3}\right)_{\downarrow d_4},
\end{gather}
such that $d = d_1 \cdot d_3 \cdot d_4, \ d_1, d_3, d_4 \in \mathbb{N}^+, d_3 \le d_{\uppsi_2}, d_3 | \frac{d}{d_1}$. The proposed downsampling scheme can be extended to an arbitrary number of scattering orders.

% Since the filterbank configuration is known, all required combinations of downsampled filters and their downsampling factors can be precomputed.




\subsection{Convolutions}
Optimal \ac{fft} convolutions can be achieved by performing downsampling in the frequency domain instead of the time domain. It is straightforward to verify that $|x * \uppsi|_{\downarrow r} = \left|(x * \uppsi)_{\downarrow r}\right|$, since the modulus is an element-wise operation, for some downsampling factor $r$. Given the signal and filter Fourier transforms $\hat{x}$ and $\hat{\uppsi}$, we then have
\begin{equation}
\label{eqn:circconv}
    x \otimes \uppsi [n] \xleftrightarrow{\mathcal{FFT}}  \hat{x} \cdot \hat{\uppsi} [k],
\end{equation}
where the $\otimes$ operator represents a circular convolution and $k$ is the frequency index.

Given that $r|N$, we can express (\ref{eqn:circconv}) when downsampled as a periodised summation \citep{waveletsandsubbandcoding} in the frequency domain
\begin{equation}
    (x \otimes \uppsi [n])_{\downarrow r} \xleftrightarrow{\mathcal{FFT}}  \frac{1}{r} \sum_{i=0}^{r-1} \hat{x} \cdot \hat{\uppsi} [k + iN/r], %https://citeseerx.ist.psu.edu/document?repid=rep1&type=pdf&doi=5216ea733e562541b33a7f97dab0de072b2e8827
\end{equation}
which can be efficiently implemented via shape manipulation of tensors in computational packages like MATLAB or PyTorch \citep{pytorch}. To compute valid convolutions, we must pad $x$ and $\uppsi$ to have a total length of $N = N_x + d + c$, where $c \in \mathbb{N}^+$ is a constant that ensures that $d | N$. 

% Note that only $d$ additional samples are required instead of $2d$, since $d$ is, by definition, the time support of $\phi$ in number of samples. Padding is applied equally to the left and right of $x$. For further reduction of boundary effects, reflection padding is utilised. Kymatio's 2D scattering implementation utilises a similar padding scheme. 

% Thus far, the techniques discussed may be equivalently implemented on many scattering transforms. Separable scattering benefits from reduced computations resulting from filter separability. In particular, given a $m$-dimensional signal $x[\vect{n}]$ indexed by $\vect{n} = (n_1, ..., n_m)^T$, with each dimension having downsample factors expressed as a vector $\vect{r} = (r_1, ..., r_m)^T$, we can chain downsampling for each dimension:
% \begin{equation}
%     \left(x * \uppsi_{\vect{\lambda}}[\vect{n}]\right )_{\downarrow \vect{r}} = \left( \left(x[\vect{\lambda}] * \uppsi_{\lambda_i}[n_i]\right)_{\downarrow r_1} ... * \uppsi_{\lambda_m}[n_m]\right)_{\downarrow r_m}.
% \end{equation}.

% We abuse notation to indicate that $\downarrow r_i$ is an operator that downsamples the $i$'th dimension, and $\downarrow \vect{r}$ downsamples all dimensions each with their correponding downsample factor in $\vect{r} \in \mathbb{N}^m$.

% For $m$-dimensional sequences of dimensional lengths $N_1, ..., N_m$, the forward and inverse \acp{fft} computation time is reduced from $2\sum_{i} T \log N_i$ (no frequency periodisation), with $T = \prod_i N_i$ to $\sum_{i} \frac{T}{K_i} \log \frac{N_i}{K_i} + P \log P$, where $K_1 = 1$, $K_i = \prod_{j < i} r_j$ and $P = \prod_i \frac{N_i}{r_i}$. However, the number of convolution multiplications in the frequency domain increases from $\prod_i N_i$ to $\sum_i \prod_j \frac{N_j}{K_i} \leq m \prod_i N_i$.

% In contrast, optimal non-separable scattering transforms can achieve a computational cost of $\sum_{i} N_i \log N_i + P \log P$, since a full forward \ac{fft} is required to compute a convolution with a non-separable $m$-dimensional filter.

% The greatest computational advantage is during direct convolution computations, as is present in \acp{cnn}. For an $m$-dimensional filter impulse response with filter lengths $L_1, ..., L_m$, the computational cost reduces from $P\prod_i L_i$ to $P \sum_i L_i$.


\section{Results}



\subsection{MNIST}

% We benchmark separable \ac{ws} against 2D scattering by repeating the MNIST hand-written digit dataset experiment in \citep{2dscattering}. 

The MNIST dataset \citep{mnist} contains 60000 training and 10000 test samples. Unlike in \citep{2dscattering}, which decorrelates scattering coefficients with a discrete cosine transform prior to classification, we perform classification on the scattering coefficients directly. Kymatio \citep{kymatio} is used to produce the 2D scattering coefficients. Our implementation of separable scattering is implemented similarly to Kymatio, with PyTorch \citep{pytorch} as a backend for FFT convolutions and \ac{nn} models.

Unless specified otherwise, all experiments have $\beta = \alpha = 2.5$. Scattering features are normalised prior to classification, according to the mean and \ac{std} calculated on the training set. No data augmentation is performed.

\begin{table}
\centering
\caption{MNIST classification error rate (\%) of separable and \ac{2d} scattering coefficients using a \ac{nn} classifier} \label{tab:mnistnn}
\begin{tabular}{|lr|l|} \hline
\multirow{2}{*}{2D WS + NN}        & $l=1$, $J=2$  & $0.64 \pm 0.05$    \\
          & $l=2$, $J=3$ & $0.50 \pm 0.03$     \\ \hline
\multirow{2}{*}{Separable WS + NN}  & $l=1$, $d=(4,4)$ & $0.63 \pm 0.05 $    \\
          & $l=2$, $d=(4,4)$  & $0.52 \pm 0.04$     \\ \hline
\end{tabular}
\end{table}


% Table \ref{tab:mnistlda} shows the results for MNIST using a \ac{lda} classifier from the Scikit-learn python package \citep{sklearn}, with a covariance shrinkage factor of $5\cdot 10^{-3}$. Scattering is performed for $l=1, 2$ levels.

% \begin{table}[!h] 
% \centering
% \caption{MNIST classification error rate (\%) of separable and \ac{2d} scattering coefficients  using an \ac{lda} classifier} \label{tab:mnistlda}
% \begin{tabular}{|r|rr|rr|} \hline
% \multirow{2}{*}{Training size}              & \multicolumn{2}{l}{2D}                               & \multicolumn{2}{l|}{Separable}                             \\
%  & \multicolumn{1}{l}{$l=1$} & \multicolumn{1}{l}{$l=2$} & \multicolumn{1}{l}{$l=1$} & \multicolumn{1}{l|}{$l=2$} \\ \hline
% 1000                              & 4.58                        & 2.69                        & 4.87                        & 4.58                        \\
% 2000                              & 3.89                        & 1.62                        & 4.31                        & 2.6                         \\
% 5000                              & 3.54                        & 1.12                        & 3.84                        & 1.7                         \\
% 10000                             & 3.35                        & 1.05                        & 3.54                        & 1.5                         \\
% 20000                             & 3.16                        & 0.87                        & 3.41                        & 1.41                        \\
% 40000                             & 2.92                        & 0.86                        & 3.34                        & 1.25                        \\
% 60000                             & 2.92                        & 0.81                        & 3.27                        & 1.34     \\ \hline                  
% \end{tabular}
% \end{table}

Due to its separability and non-angularly spaced filters, separable scattering does not perform as well compared to 2D scattering when using simple classifiers, such as \ac{lda} \citep{lda}. To illustrate that this performance discrepancy is not of significant consequence in a \ac{nn} setting, we test performance on the full dataset utilising a simple architecture. The neural network architecture used has an input layer with 256 neurons, followed by two hidden layers with 128 and 64 neurons respectively. The output layer has 10 neurons - one for each digit. Input and hidden layers are followed by a batch norm layer \citep{batchnorm} and ReLU activation function \citep{relu}. The output layer is followed by a softmax function. The Adam optimiser \citep{adam} is used with cross-entropy loss, a batch size of 256 and learning rate of $3 \cdot 10^{-5}$. 5000 of the 60000 training samples are reserved for validation and removed from the training set. Training is stopped when validation loss starts to increase. The \ac{nn} model is initialised with random weights, and the experiment is repeated 50 times. Different invariance scales were tested, and the best results are reported in table \ref{tab:mnistnn}. Tests are repeated for $l \in \{1, 2\}$ levels of scattering.





\subsection{MedMNIST3D}


The MedMNIST3D datasets are a subset of the MedMNIST dataset group \citep{medmnist}, where each 3D dataset contains $28\times 28 \times 28$ images with 2, 3 or 11 classes. Train, test and validation data partitions are provided by the authors. All datasets have on the order of 1000 training samples. We compare the baseline \ac{nn} results provided in \citep{medmnist} with separable scattering features combined with a simple \ac{nn} classifier.

We use \ac{nn} classifier with an input layer containing 1024 neurons, followed by two hidden layers with 512 and 256 neurons respectively. Input and hidden layers layers are each followed by a batch norm layer \citep{batchnorm} and a ReLU non-linearity \citep{relu}. For datasets with two classes, the output layer is a single neuron followed by a sigmoid activation function. For datasets with more than two classes, the output layer has a size equal to the number of classes, followed by a softmax activation. Binary cross-entropy loss are used for datasets with two classes, otherwise cross-entropy loss is used. The Adam optimiser \citep{adam} with a learning rate of $1 \cdot 10^{-5}$ is used. All other configuration parameters are identical to the model used for the MNIST dataset.

A single level of scattering coefficients are computed, with $Q = (2, 2, 2)$ and $d = (4, 4, 4)$. Many MedMNIST3D datasets tend to be unbalanced, implying that \ac{auc} is a more reliable metric to measure model performance. The results are shown in table \ref{tab:medmnist}, in which \ac{sota} \ac{auc} performance is achieved for the Organ, Adrenal and Vessel datasets. Table \ref{tab:medmnist} indicates the number of classes ($c$) for each of the datasets, with accuracy (ACC) also shown for reference. The performance of our method on non-\ac{sota} results are comparable to the other baseline \ac{nn} approaches presented in \citep{medmnist}. It is likely that better results can be achieved by the proposed method if the filters are made learnable and/or scattering parameters are uniquely optimised for each dataset.

\begin{table*}[t!]
    \centering
    \caption{MedMNIST classification results of compared to baseline \ac{nn} approaches (\citep{medmnist})} \label{tab:medmnist}
    \begin{adjustbox}{angle=90}
\begin{tabular}{|l|rr|rr|rr|rr|rr|rr|}
\hline
\multirow{2}{*}{Methods}                & \multicolumn{2}{c}{Organ ($c=11$)}                  & \multicolumn{2}{c}{Nodule ($c=2$)}                 & \multicolumn{2}{c}{Fracture ($c=3$)}               & \multicolumn{2}{c}{Adrenal ($c=2$)}                & \multicolumn{2}{c}{Vessel ($c=2$)}                 & \multicolumn{2}{c|}{Synapse ($c=2$)}                \\
                         & \multicolumn{1}{l}{AUC} & \multicolumn{1}{l}{ACC} & \multicolumn{1}{l}{AUC} & \multicolumn{1}{l}{ACC} & \multicolumn{1}{l}{AUC} & \multicolumn{1}{l}{ACC} & \multicolumn{1}{l}{AUC} & \multicolumn{1}{l}{ACC} & \multicolumn{1}{l}{AUC} & \multicolumn{1}{l}{ACC} & \multicolumn{1}{l}{AUC} & \multicolumn{1}{l|}{ACC} \\ \hline
ResNet-1810+2.5D        & 0.977                   & 0.788                   & 0.838                   & 0.835                   & 0.587                   & 0.451                   & 0.718                   & 0.772                   & 0.748                   & 0.846                   & 0.634                   & 0.696                   \\
ResNet-1810+3D          & 0.996                   & 0.907                   & 0.863                   & 0.844                   & 0.712                   & 0.508                   & 0.827                   & 0.721                   & 0.874                   & 0.877                   & 0.82                    & 0.745                   \\
ResNet-1810+ACS41       & 0.994                   & 0.900                     & 0.873                   & 0.847                   & 0.714                   & 0.497                   & 0.839                   & 0.754                   & 0.930                    & 0.928                   & 0.705                   & 0.722                   \\
ResNet-5010+2.5D        & 0.974                   & 0.769                   & 0.835                   & 0.848                   & 0.552                   & 0.397                   & 0.732                   & 0.763                   & 0.751                   & 0.877                   & 0.669                   & 0.735                   \\
ResNet-5010+3D          & 0.994                   & 0.883                   & 0.875                   & 0.847                   & 0.725                   & 0.494                   & 0.828                   & 0.745                   & 0.907                   & 0.918                   & \textbf{0.851}          & \textbf{0.795}          \\
ResNet-5010+ACS41       & 0.994                   & 0.889                   & 0.886                   & 0.841                   & \textbf{0.750}           & \textbf{0.517}          & 0.828                   & 0.758                   & 0.912                   & 0.858                   & 0.719                   & 0.709                   \\
auto-sklearn11          & 0.977                   & 0.814                   & \textbf{0.914}          & \textbf{0.874}          & 0.628                   & 0.453                   & 0.828                   & \textbf{0.802}          & 0.910                   & \textbf{0.915}          & 0.631                   & 0.730                    \\
AutoKeras12             & 0.979                   & 0.804                   & 0.844                   & 0.834                   & 0.642                   & 0.458                   & 0.804                   & 0.705                   & 0.773                   & 0.894                   & 0.538                   & 0.724                   \\ \hline
Separable WS + NN (Ours)  & \textbf{0.998}          & \textbf{0.941}          & 0.858                   & 0.797                   & 0.614                   & 0.458                   & \textbf{0.875}          & 0.792                   & \textbf{0.962}          & 0.895                   & 0.715                   & 0.525    \\ \hline              
\end{tabular}
\end{adjustbox}
\end{table*}



\chapter{Conclusion}

In this dissertation, we investigate the viability of wavelet transforms as alternative \ac{tf} decompositions in for detecting and classifying whale calls. 

In chapter \ref*{chap:p1}, we show that the \ac{cwt} can be used for \ac{se} detectors, which can, under certain conditions, yield marginally better performance in terms of time-localisation compared to the \ac{stft}. Median filtering of the \ac{se} statistic and k-means clustering further improve the interpretability and accuracy of the \ac{se} detector. We show that this detector is much better than \acp{bled}, which are the most commonly used signal detectors in \ac{pam}.

Chapter \ref*{chap:p2} further expands on the ideas of chapter \ref*{chap:p1}, by including additional time-shift invariance to the \ac{tf} decomposition via first-level wavelet scattering coefficients. We additionally model the entropy as a \ac{gmm}, which is a further improvement of the k-means implementation in the first paper. We lay the groundwork for efficient downsampling strategies in for wavelet scattering computations. Since wavelet scattering can be used a features, we combine the detector output with a classifier to investigate the practicality of such a system. We provide critical discussions on the dataset that is used, and highlight the real-world problems faced in evaluation \ac{pam} detections and classification systems.

From the foundation established in chapter \ref*{chap:p2} and inspired by joint-\ac{tf} scattering, we generalise the notation of using separable wavelet filters in a multi-dimensional scattering transform in chapter \ref*{chap:p3}. We show that, in a modern \ac{ml}-setting, it can be equally effective compared to other wavelet scattering filterbanks, yet provides significant advantages due to filter separability, which includes greater configuration freedoms and computational advantages. 

Further work includes deploying separable wavelet scattering and the improved \ac{se} detector from chapter \ref*{chap:p2} in a \ac{pam} detection and classification system, while evaluating its performance compared to \ac{nn}-based approaches.

The papers presented in this dissertation naturally introduce wavelet scattering to both signal detection and classification for \ac{pam} systems, thereby addressing some of the shortcomings of \acp{nn}. We develop a new type of scattering which is particularly well suited to the requirements of \ac{pam}.
% \end{singlespace}

\bibliographystyle{stb-bib-eng-a}
\bibliography{references_consolodated}


\end{document}
