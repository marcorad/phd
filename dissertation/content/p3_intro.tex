\chapter{Wavelet Scattering in Higher Dimensions}
\label{chap:p3i}

In chapter \ref*{chap:p2i}, we reviewed a generalised definition of a multi-dimensional TF decomposition $X$, which is processed by a set of invariance-enforcing filters $\Phi_i$. Wavelet scattering in 1 dimension utilises a single invariance-enforcing filter, $\phi$, as discussed in chapter \ref*{chap:p2}.

Previous chapters have been specifically targeted at underwater bioacoustic applications. We deviate from specific applications in chapter \ref*{chap:p3} in order to provide a generalised definition of a new type of scattering. This is considered to be the main and largest contribution of this dissertation. The new proposed scattering is intended for use in multiple applications which can benefit from filter separability properties (see section \ref*{sec:p3i:sep}).

Wavelet scattering filterbanks have been extended to multiple dimensions using various different wavelets and filterbank construction techniques. This chapter provides an over\-view of the existing wavelet filterbanks and filters and their properties. We then identify a gap in the literature, which gives rise to a generalisation of cascasded 1D scattering transforms -- separable wavelet scattering.

\section{Types of Scattering Transforms}
Existing scattering transforms are listed and described below for convenience:
\begin{enumerate}
    \item 1D scattering \citep{1dscattering1, ws}. This is the first-introduced scattering transform, which serves as a generalisation of MFCCs. It improves on MFCCs by recovering the information lost due to invariance filter (averaging) operations. Morlet wavelets are used, with cascaded scattering levels having its own number of filter per octave.
    \item 2D scattering \citep{2dscattering}. Gabor filters modified to have 0-mean are used to construct the filterbank. The filterbank consists of exponentially placed filters in one axis, which is then rotated by equally spaced angles to cover the frequency plane. In particular, in polar coordinates $(r, \theta)$ in the frequency domain, filters are placed exponentially along $r$ with a single filter per octave, but linearly along $\theta$.
    \item Harmonic scattering \citep{harmonicscattering} (3 dimensions). The spherical harmonics are used as filters which consitute an orthogonal basis in 3 dimensions. These filters have ``lobes'' that extend in the along various directions in the frequency domain, depending on the harmonic.
    \item 3D scattering -- extension of 2D scattering \citep{3dscattering}. The filterbank is constructed similar to 2D scattering. In spherical coordinates $(\rho, \theta, \phi)$ in the frequency domain, filters are spaced exponentially along $\rho$ with a single filter per octave, and linearly along $\theta$ and $\phi$.
    \item Joint time-frequency \citep{ws_joint_tf,jointtfscattering2} scattering -- 2D filters on 1D scattering scalogram. A first level of scattering is performed on a 1D time signal to produce a 2D scalogram ($X[k, m]$). 2D scattering is then performed on subsequent levels, with $X[k, m]$ as input. The 2D wavelets are separable - a product of 1D wavelets. This allows 1D scattering on audio to have features that are also dependent on adjacent frequency bins, which is not possible with 1D scattering only. The extracted features can therefore correlate with frequency modulations while retaining the benefits from 1D. It is conceptually similar to a \ac{cnn} operating on a \ac{stft} spectrogram. In the frequency domain with axes $(\omega_1, \omega_2)$, the 2D filters are constructed as the cross-product between filter sets that placed exponentially in both $\omega_1$ and $\omega_2$.
    This is the only literature considering separable filters.
\end{enumerate}

\section{Benefits of Separable Filters}
\label{sec:p3i:sep}

Recently, wavelet scattering operations implemented as \acp{cnn}, referred to as a scattering network, have been of increasing interest in order to increase the effectiveness of \ac{cnn} filters \cite{2dscattering, scattering_birdsong}. Results indicate that making some aspects of scattering networks learnable yields significant improvements compared to non-learnable filters. Particularly, fixed spatial and/or time filters with learnable across-channel filters seem to obtain the best results \cite{nnphasecollapse}.

For gradient descent, it may be beneficial to perform convolutions directly, and not in the Fourier domain. Separable filters have been shown to be effective in a \ac{cnn} settings, while also allowing for much quicker convolution operations \cite{separablecnn,separablecnn2} when computed directly. Separable filters also allow for the independance of each dimension, which can be particularly useful for optimisations such as downsampling. This type of scheme is illustrated by higher-dimensional dyadic wavelet transforms such as the \ac{dwt} \cite{waveletbook}.

Downsampling schemes are trivial when the axes are independent -- a filterbank can be configured independently for each axis, where subsequent axes remain oblivious to previous operations. When a mix of time and spatial dimensions are present, existing techniques such as 2D, 3D and harmonic wavelet scattering may not be suitable, since their invariance filters cannot provide a certain amount of invariance for each dimension independently, due to their filterbank construction. Additionally, 2D and 3D scattering traditionally do not allow the user to create a filterbank with more than 1 filter per octave across $r$ and $\rho$ respectively. Signals with different time/spatial supports (i.e., a signal may be much longer in one dimension compared to another) will likely require different invariance scales across dimensions. As such, conventional higher dimensional scattering transforms may not be appropriate. 

The type of scattering transform used is usually highly dependent on the signal dimen\-sionality -- there is no unifiying multi-dimensional analogy of 1D scattering as is present with dyadic wavelet transforms \citep{waveletbook}. We construct this generalisation in the chapter \ref*{chap:p3} via separable filters.


