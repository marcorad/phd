\chapter{Conclusion}

In this dissertation, we investigate the viability of wavelet transforms as alternative \ac{tf} decompositions in for detecting and classifying whale calls. 

In chapter \ref*{chap:p1}, we show that the \ac{cwt} can be used for \ac{se} detectors, which can, under certain conditions, yield marginally better performance in terms of time-localisation compared to the \ac{stft}. Median filtering of the \ac{se} statistic and k-means clustering further improve the interpretability and accuracy of the \ac{se} detector. We show that this detector is much better than \acp{bled}, which are the most commonly used signal detectors in \ac{pam}.

Chapter \ref*{chap:p2} further expands on the ideas of chapter \ref*{chap:p1}, by including additional time-shift invariance to the \ac{tf} decomposition via first-level wavelet scattering coefficients. We additionally model the entropy as a \ac{gmm}, which is a further improvement on k-means. We lay the groundwork for efficient downsampling strategies in for wavelet scattering computations. Since wavelet scattering can be used a features, we combine the detector output with a classifier to investigate the practicality of such a system. We provide critical discussions on the dataset that is used, and highlight the real-world problems faced in evaluation \ac{pam} detections and classification systems.

From the foundation established in chapter \ref*{chap:p2} and inspired by joint-\ac{tf} scattering, we generalise the notation of using separable wavelet filters in a multi-dimensional scattering transform in chapter \ref*{chap:p3}. We show that, in a modern \ac{ml}-setting, it can be equally effective compared to other wavelet scattering filterbanks, yet provides significant advantages due to filter separability, which includes greater configuration freedoms and computational advantages. 

Further work includes deploying separable wavelet scattering and the improved \ac{se} detector from chapter \ref*{chap:p2} in a \ac{pam} detection and classification system, while evaluating its performance compared to \ac{nn}-based approaches.

The papers presented in this dissertation naturally introduce wavelet scattering to both signal detection and classification for \ac{pam} systems thereby addressing some of the shortcomings of \acp{nn}. We develop a new type of scattering which is particularly well suited to the requirements of \ac{pam}.